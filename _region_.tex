\message{ !name(../PEI.tex)}%
% Modelo para relat￳rio da disciplina de Projecto de Engenharia Informatica
% do MEI.
%
% Incorpora elementos impostos pelo Regulamento de Estudos Pos-Graduados da
% Universidade de Lisboa (Deliberacao 1506/2006 - Diario da Repblica, 2.a s←rie 
% - n.o 209 - 30 de Outubro de 2006)
%
\documentclass[12pt,openright,twoside]{report}
\usepackage[show]{chato-notes}
\usepackage[utf8]{inputenc}
% Quem tiver problemas com os acentos, trocar utf8 por latin1

\usepackage[portuguese,english]{babel}
\usepackage{times}
\usepackage{url}
\usepackage{graphicx}
\usepackage{mdwlist}
\usepackage[nottoc]{tocbibind}
\usepackage{csquotes}
\usepackage[table,hyperref,x11names]{xcolor}
\usepackage{array,booktabs}
\usepackage{multirow}

\usepackage{dialogue}
%To get figures and tables side by side
\usepackage{floatrow}
\newfloatcommand{capbtabbox}{table}[][\FBwidth]
% end figures and tables side by side

%footnote in tables 
\usepackage{threeparttable}
%end footnote in tables

% Indice remissivo
\usepackage{makeidx}
\makeindex

%quotes
\usepackage{epigraph}
\usepackage{attrib}
\usepackage{titlesec}

%\titlespacing{\subsubsection}{0pt}{0pt}{0pt}
\titleformat{\subsubsection}[runin]{\normalfont\bfseries}{\thesubsection.}{3pt}{}


\usepackage[nomain,acronym,xindy,nonumberlist]{glossaries}
\newacronym{sdn}{SDN}{Software Defined Network}

\newacronym{nib}{NIB}{Network Information Base}

\newacronym{wan}{WAN}{Wide Area Network}

\newacronym{nos}{NOS}{Network Operating System}

\newacronym{os}{OS}{Operating System}

\newacronym{onf}{ONF}{Open Network Foundation}

\newacronym{of}{OF}{OpenFlow}

\newacronym{fifo}{FIFO}{First In,First Out}

\newacronym{smr}{SMR}{State Machine Replication}

\newacronym{arp}{ARP}{Address Resolution Protocol}
\newacronym{api}{API}{Application Programming Interface}
\newacronym{lru}{LRU}{Least Recently Used} 
\newacronym{ip}{IP}{Internet Protocol}
\newacronym{icmp}{ICMP}{Internet Control Message Protocol} 
\newacronym{ipv6}{IPv6}{Internet Protocol version 6}
\newacronym{vlan}{VLAN}{Virtual Local Area Network} 
\newacronym{mac}{MAC}{Media Access Control}


\newacronym{rest}{REST}{Representational State Transfer} 
\newacronym{vip}{VIP}{Virtual Enpoint Internet Protocol} 
\newacronym{rpc}{RPC}{Remote Procedure Call} 
\newacronym{sql}{SQL}{Structured Query Language} 

\newacronym{tcp}{TCP}{Transmission Control Protocol}
\newacronym{rcp}{RCP}{Routing Control Platform}
\newacronym{bgp}{BGP}{Border Gateway Protocol}
\newacronym{as}{AS}{Autonomous System}
\newacronym{ibgp}{iBGP}{interior Border Gateway Protocol}
\newacronym{url}{URL}{Uniform Resource Locator}
\newacronym{dhcp}{DHCP}{Dynamic Host Configuration Protocol}
\newacronym{http}{HTTP}{HyperText Transport Protocol} 
\makeglossary  

% Links
\usepackage{hyperref}

% Package para cabecalhos
\usepackage{fancyhdr}
\usepackage{lastpage}

\usepackage[font={small}]{caption}
\usepackage{subcaption}
\usepackage[sortcites=true, firstinits=true, isbn=false,
url=false, doi=false, eprint=false]{biblatex}

\bibliography{bibliografia,web}
\fancyhf{} %
\lhead{\nouppercase {\leftmark}} %
\rhead{\nouppercase {\bf \thepage}}
\renewcommand{\headrulewidth}{0.1pt}

% Comando para inserir pagina em branco (inserida na numeracao, mas sem
% numero impresso) para quando e' preciso obrigar um capitulo a comecar
% do lado direito (pagina impar)
\newcommand{\LIMPA}{
\newpage
\mbox{}
\thispagestyle{empty}
}

% Igual, mas insere pagina com numero impresso (normalmente nao se usa)
\newcommand{\LIMPAC}{
\newpage
\mbox{}
\thispagestyle{plain}
}


%%%%%%% PERSONAL COMMANDS %%%%%%%%%%%
\newcommand{\distcontrollers}{\cite{:vn, Tootoonchian:2010vy,Koponen:2010th,Yeganeh:2012jm}}
\newcommand{\distcontrollerspaper}{\cite{Tootoonchian:2010vy, Koponen:2010th,Yeganeh:2012jm}}
%END 

\newcommand{\tbl}[2]{\begin{tabular}{#1}#2\end{tabular}}
\newcommand{\ml}[2]{#1 $\pm$ #2} 

%
% ALTERAR AQUI AS INFORMACOES RELATIVAS AO PROJECTO
%
\newcommand{\PEITITULO}{A Consistent and Fault-Tolerant Network
  Information Base for Scalable Software Defined Networks}
\newcommand{\PEIAutor}{Fábio Andrade Botelho}
\newcommand{\PEIAutorNumAluno}{41625}

%Orientador e CoOrientador *sem* titulos (e.g. Prof. Doutor)
\newcommand{\PEIOrientador}{Alysson Neves Bessani}
\newcommand{\PEICoOrientador}{Fernando Manuel Valente Ramos} %se nao se aplicar, nao importa o que aqui esteja

%Se aplicavel, o supervisor pode ter um titulo (Dr., Eng.) colocado aqui
\newcommand{\PEISupervisorInstituicao}{Nome Completo do Supervisor}  %se nao se aplicar, nao importa o que aqui esteja

\newcommand{\PEIAnoLectivo}{2012/2013}
\newcommand{\PEIAno}{2013}

% Comentar/descomentar conforme conveniente
\newcommand{\PEITIPO}{DISSERTA\c{C}\~{A}O }
%\newcommand{\PEITIPO}{PROJECTO }

% Comentar/descomentar conforme conveniente
%\newcommand{\PEIIdiomaTese}{\selectlanguage{portuguese}}
\newcommand{\PEIIdiomaTese}{\selectlanguage{english}}

% Comentar/descomentar conforme conveniente
%\newcommand{\MEIEspecializacao}{Segurança Informática}
%\newcommand{\MEIEspecializacao}{Arquitectura, Sistemas e Redes de Computadores}
%\newcommand{\MEIEspecializacao}{Interac��o e Conhecimento}
%\newcommand{\MEIEspecializacao}{Engenharia de Software }

\usepackage{ifpdf}
\ifpdf
\pdfinfo {
	/Author (\PEIAutor)
	/Title (Projecto em Segurança Informatica)
	/Subject (Segurança Informatica)
	/Keywords (state machine replication, software defined networks)
	/CreationDate (D:20100510104905)
}
\fi

\usepackage[dvips]{geometry}
\geometry{a4paper=true,portrait=true,left=3cm,right=3cm,top=2.5cm,bottom=3.5cm}

\title{\PEITITULO}
\author{\PEIAutor}
%\date{\today}

\begin{document}

\message{ !name(tex/feasibility.tex) !offset(-147) }
%!TEX root = ../PEI.tex
\label{sec:feasibility:apps}
\glsresetall

We are considering applications in isolation. DeviceManager could have
applications registered on device events (updates, deletion etc.,)
that would cause more traffic to the data store. 
 


Mininet  Hi-Fi \cite{Handigol:2012tg}\footnote{Version 2.0 (mininet-2.0.0-113012-amd64-ovf) available at \url{http://mininet.org}}. Unfortunely we had to hack the box configuration to avoid that hosts  triggered gratioutious data packets that would  trigger traffic to the controller when initializing the switch-controller connection. We changed the \texttt{net\.ipv6\.conf\.all\.disable\_ipv6,net\.ipv6\.conf\.default\.disable\_ipv6} fields in the file \texttt{/etc/sysctl\.conf} to true\footnote{This is a known problem https://mailman.stanford.edu/pipermail/mininet-discuss/2010-November/000167.html (contains solution)}. 

We do not log creation, and deletion of tables. Those will not appear in the workloads. This is nor relevant since it usually happens upon a first switch connection or first host packet processed by an application  for that switch


\section{Learning Switch}
\label{sec:feasibility:ls}
\glsresetall
%%%%%%%%%%%%%%%%%%%%%%%%%%%%%%%%%%%%%%%%%%%%%%%%%%%%

\todo{IT IS NOT IP. IT IS A MAC.}
\todo{Learning switch is what Kandoo calls a local app}
\begin{figure}[ht]

  \begin{subfigure}[b]{0.5\textwidth}
                \centering
                \includegraphics[width=\textwidth]{pic/feasibility/ls-events-broadcast}
                \caption{Broadcast packet.}
                \label{fig:ls:interaction:broadcast}
        \end{subfigure}%
        ~
        \begin{subfigure}[b]{0.5\textwidth}
                \centering
                \includegraphics[width=\textwidth]{pic/feasibility/ls-events-unicast}
                \caption{Unicast packet.}
                \label{fig:ls:interaction:unicast}
        \end{subfigure}
        \caption[Learning Switch workloads]{Broadcast packets trigger a write for the source address of the respective packet. Unicast packets have to additionally read the source address port location.}
        \label{fig:ls:interaction}
\end{figure}

The Learning Switch application emulates the hardware layer 2 switch
forwarding process. For each switch a different \emph{\texttt{MAC}-to-switch-port}
table is maintained in the data store. Each table is populated using
the source address information (i.e., \texttt{MAC} and port)  present in every OpenFlow
\texttt{packet-in} request for the purpose of maintaining the location
of devices. After learning this location, the controller can install
rules in the switches to forward packets from a source to a
destination. Until then, the controller must instruct the switch to
\emph{flood} the packet to every port, with the exception of
  the ingress port. Despite being distributed, each switch-table is
  only accessed by the controller managing the switch in
  question. Even so, we justify the study  of this application for two
  reasons: (i) it benefits from the fault-tolerance property of
  our distribution and (ii) it is commonly used as the
  single-controller benchmark application in the literature \cite{Tootoonchian:2012uia}. 

\todo{to be common you need more than 1 example}

Figure \ref{fig:ls:interaction}  shows the detailed interaction between the
switch, controller (Learning Switch) and data store for two possible
cases. First (figure \ref{fig:ls:interaction:broadcast}), the case for broadcast packets that require
one write operation to store the switch-port  of the
source address. Second (figure \ref{fig:ls:interaction:unicast}),   the case for unicast
packets, that not only stores source information, but also read the
switch egress port for the destination address.  


\note{Nao esta  claro que a tabela nesta app e single
  reader, single writer. NInguem entendeu...}

Original:  Map<IOFSwitch, Map<MacVlanPair,Short>> . Modified : Map<IOFSwitch, Map<MacVlanPair,Short> . Originally instanced with concurrent hashmaps for the main map and each table map. 
CHANGE: We still maintain a concurrentMap for the main switch-to-KeyValueTable map. This is required since the manipulation of the map is done concurrenty by: each switch thread and  by the REST API.
There isn't any problem since KeyValueTable is  thread safe \ref{section.datastore.thread.safety}

Original: It is critical that the size of each switch table is limited 
due to resource usage.  Mac addresses have to be recycled as devices enter and leave the network. Or simple as traffic dynamics change. Either way in some networks we can't possibly support all the hosts \ref{sec.learning.switch.study.size} for table. Remember each table can contain all hosts in the network! Traditionally the LearningSwitch uses a LRULinkedHashMap (Least Recently Used Linked Hash Map). This imposes a limit on the number of entries present in the table. The behaviour of this LRULInkedHashMap is to remove the oldest entry in the table whenever the threshold is reached.  The default threshold (used in all our experiences unless we state otherwise) is of 1000 hosts per table. (Remember that each table can potentially keep an entry for each host in the all network! ). 
Change:  We replicate this kind of table in the data store. Which was fairly simple. Off course the behaviour changes in the case we use a cache. As discussed in section \ref{sec.learning.switch.lru.cache}

The controller only replies to one switch.  This causes the problem set by \cite{of.cpp} of mac broadcasting REVISE EXAMPLE.  The path is set switch by switch. 

The application is configured with a idle timeout of 5 and a hard timeout of 0 seconds. As such the switch deletes the entry after 5 seconds of inactivity. This will result in a delete from the switch table that will trigger an OpenFlow \texttt{FLOW\_REMOVED } message from the switch to the controller. The Learning switch applications will process this OF message and delete the corresponding entry in the data store. Then, (immediately after) it will instruct the switch the remove the reverse entry from its table. The switch upon processing this message will trigger another \texttt{FLOW\_REMOVED} message to the controller.  This recursive process is not infinitive. The switch will only trigger \texttt{FLOW\_REMOVED} messages when it deletes an entry from the table. If the controller instructs it to remove something that isn't there, we will not trigger any message (sec. 6.4 of \cite{openflow-spec}). 


\subsection{Broadcast Packet}
This workload corresponds to the operations performed in the data
store when processing broadcast packets in a OpenFlow
\texttt{packet-in} request.  Table \ref{table:lsw0:broadcast} shows that for the
purpose of associating the source address of the packet to the ingress
switch-port where it was received, the Learning Switch application performs one
write operation with a request size of 113 bytes and reply size of 1
byte. 

\begin{table}[ht]
\centering 
\begin{tabular}{l c c c c}
 Operation & Type & Request & Reply \\ \toprule 
 Associate source address to ingress port & W & 113 & 1 \\ \bottomrule
\end{tabular}
\caption[Workload lsw-0-broadcast( Broadcast Packet) operations]{Workload lsw-0-broadcast( Broadcast Packet) operations and sizes (in bytes).}
\label{table:lsw0:broadcast}
\end{table}

\subsection{Unicast Packet}
Workload \textbf{lsw1-1} is the result of processing an OpenFlow request
triggered by an unicast packet. Thus,  when compared to the previous
workload (\textbf{lsw1-0} covering broadcast packets), an additional
operation is required to query the switch-port location of the
destination address. Table \ref{table:lsw0:unicast} provides a summary all the data
store operations in this workload. 

\begin{table}[ht]
\centering 
\begin{tabular}{l c c c c}
 Operation & Type & Request & Reply  \\ \toprule 
 Associate source address to ingress port & W & 113 & 1\\\midrule
Read egress port for destination address & R & 36 & 77 \\\bottomrule
\end{tabular}
\caption[Workload lsw-0-unicast( Unicast Packet) operations]{Workload lsw-0-unicast( Unicast Packet) operations and sizes (in bytes).}
\label{table:lsw0:unicast}
\end{table}

\subsection{Optimizations}
Our serialization process between ewsdn and now has changed. In ewsdn we used longer tablenames and java serialization. Aftewards we changed our codes so that each tablename is the concatenation of ``LS''  and the switch identifier (which can be as long as FIXME characters.  Also we have manually convert the information stored in the data store. Notice for example that between the examples the read egress port operation, the return value (a switch port idenfifiers) varies is 12 times greater in the java serialization process. 


\begin{table}[ht]
\centering 
\begin{tabular}{l c c c c}
 Operation & Type & Request & Reply \\ \toprule 
Associate source address to ingress port & W & 29 & 1\\\midrule
Read egress port for destination address & R & 27 & 6 \\\bottomrule
\end{tabular}
\caption[Workload lsw-1-unicast( Unicast Packet) operations]{Workload lsw-1-unicast( Unicast Packet) operations and sizes (in bytes).}
\label{table:lsw1:unicast}
\end{table}

\begin{figure}[ht]
\centering
\includegraphics[scale=0.5]{../data/reportGenerator/lsw-0-broadcastlsw-0-unicastlsw-1-broadcastlsw-1-unicasttxLatCmp.pdf}
\caption[Learning Switch workloads performance comparison]{Learning Switch workloads performance comparison (90th percentile). }
\end{figure}
\subsection{Cache}

\label{sec.learning.switch.lru.cache} Discuss cache implications of
least recently used. 

We only get values from cache if they exist despite the time they are
present there. This happens since Learning Switch installs rules with
a idle timeout. So when rules expire, the switch warns the Learning
Switch application which in turn deletes the entries in the data
store. So we don't actually require to be worried about stale values
in the cache being invalid (e.g., a host moved from one port to
another). If they are, the idle mechanism associated with rules
installed in the data plane will, eventually, correct the situation.
Off course we are not strongly consistent anymore. 

We don't actually improve on the micro-benchmarks tested measures
shown throughout this chapters. We do not improve simply because with
cache we do not avoid or improve any of the data store interactions
present in table \ref{table:work:lsw1-1} (that shows the latest
learning switch workload).  With cache we will only improve on the
long run, since we can now avoid the two type of requests present in
that table. First we can avoid re-writing the source address to source
port association when we already now it. In the original learning
switch association this is re-write is not costly ($\Omega(1)$ ) and also
has the functional impact of refreshing the entry timestamp such that
the least recently used table can keep up consistently with the last
active hosts and delete the not active ones. In the cache
implementation this is not true anymore. Now, the active hosts
actually get forgotten somewhere in time as newly (unknown) entries
are added to the data store. This is not problematic since the host,
being active, will benefit from latency a lot before actually being
erased from the  the data store due to the newly added hosts. This off
course is explicitly dependant of the maximum number of entries per
table. 
\footnote{\url{http://docs.oracle.com/javase/6/docs/api/java/util/LinkedHashMap.html}}

(Note that insertion order is not affected if a key is re-inserted
into the map. (A key k is reinserted into a map m if m.put(k, v) is
invoked when m.containsKey(k) would return true immediately prior to
the invocation.)).  So NOTHING I SAID WAS TRUE! :) BUT IT APPLIES TO
GETS : In access-ordered linked hash maps, merely querying the map
with get is a structural modification.)
FIXME : YOU MUST ACTUALLY CONFIRM ALL THIS. AND ADD SUPPORT IN THE
IMPLEMENTATION OF THE LEARNING SWITCH. 

When avoiding this write
in the cache implementation we must actually be sure that we only
avoid to write to the data store when the association is known in
cache and it is actually correct (the ingress port is the same from
the packet being processed). 

The second avoidance is the read operation that queries for the egress
port of the current processed packet. We do not actually need to read
from the data store it the entry is present in the cache. First the
data is not been modified by any other controller since we are the only
ones which manipulate our switch tables. 


With cache we no longer read from the database. We do not need since
put updates the cache. So if it isn't on the cache it is not in the
data store.



% \begin{figure}
% \centering
% \includegraphics[width=\textwidth]{pic/feasibility/ls-events-unicast}
% \caption[]{\textbf{}}
% \end{figure}

\label{cenas}

% \begin{figure}
% \centering
% \includegraphics[width=\textwidth]{pic/feasibility/ls-events-broadcast}
% \caption[]{\textbf{}}
% \end{figure}

\section{Load Balancer}
\label{sec:feasibility:lb}
\glsresetall
%%%%%%%%%%%%%%%%%%%%%%%%%%%%%%%%%%%%%%%%%%%%%%%%%%%%%%%%%%%%%%%%%%%%%%%%%%%%%%%%%%%%%%%%%%%

The Load Balancer application employs a round-robin algorithm to distribute the
requests addressed to a \gls{vip} . In order to
understand its behaviour we begin by the data model currently used. Figure
\ref{fig:lb-model} shows the three different entities used in the Load
Balancer. The  VIP entity .
representss   a virtual endpoint with a specified \gls{ip}, port and
protocol address. Each VIP can be associated with one or more pools
of 
servers. Given that the distribution algorithm is round-robin, each pool
has a current assigned server (\texttt{current-member} attribute in the figure). Finally, the third entity --- Member
--- represents a real server. Each of those entities, corresponds
to a different tablee  in the data store, indexed by the entity
key attributes represented in the figure (in bold). Moreover, a fourth table is
required to associate \gls{ip} addresses to VIP resources. In light of
this data model, the load balancer logic requires the following
operations from the data store: (i) check if the source address is
associated with a VIP resource; (ii) if so, read the VIP, Pool and
Member information required to install flows in the switch and (iii)
update the pool \texttt{current-member} attribute. This description corresponds to the case where OpenFlow
\texttt{packet-in} requests are indeed addressed at a VIP
resource. The respective workload which, \textbf{markedly - replace,
  aly says is wrong TODO} is the heavier in
the Load Balancer application, is described in detail in the workload
\textbf{lbw1-0} section ahead. Alternatively, workload
\textbf{lbw1-1} captures the workload created by every
packet unrelated to a VIP. Finally, workload \textbf{lbw1-2}
considers the special case of ARP requests questioning the hardware
address of a VIP \texttt{IP}.

\begin{figure}[H]
\centering 
\includegraphics[scale=0.6]{./pic/feasibility/lb-model.pdf}
\caption{\small Simplified Load Balancer entities data model. The data
store contains a table for each entity, indexed by their keys (represent as bold attributes). }
\label{fig:lb-model}
\end{figure}

We can see the model. FIXME (take from final.tex) 
Initially we also found different index tables:  (FIXME How should i
present this kind of stuff? show the model, and then show a table in
the same figure if possible. The table should show the indexes listed
with names, types and simple explanation.) 
\texttt{vipIpToId} to map Ip (integer) into VIPID (string) 
\texttt{vipIpToMac} to map Mac (MacAddress) into VIPID (string) 
\texttt{memberIpToId} to map Ip (integer) into Member ID. 
\texttt{clientToMember} to map IPClient into Member  Remember
connected clients. 




\begin{figure}
  \centering
  \begin{subfigure}[b]{0.5\textwidth}
                \centering
                \includegraphics[width=\textwidth]{pic/feasibility/lb-events-broadcast}
                \caption{ARP packet address at a VIP.}
                \label{fig:lb:interaction:arp2Vip}
        \end{subfigure}%
        ~
        \begin{subfigure}[b]{0.5\textwidth}
                \centering
                \includegraphics[width=\textwidth]{pic/feasibility/lb-events-unicast}
                \caption{}
                \label{fig:lb:interaction:ip2Vip}
        \end{subfigure}
        \caption[Load Balancer workload events]{A \texttt{\gls{arp}} request message addressed at a VIP \gls{ip} that results in a direct \gls{arp} reply. On the left a normal \gls{ip} packet addressed at VIP should be resolved (who is responsible) and replied by installing the appropriate rules}  
        \label{fig:lb:interaction}
\end{figure}

\subsection{Packets to a VIP}
When the Load Balancer  receives a data packet addressed
at a VIP, it triggers the operations seen in table
\ref{table:workload:lbw1-0}. 
The first operation fetches a VIP resource associated with the
destination \texttt{IP} address of the packet.
If it succeeds (reply different from 0), then it proceeds to read 
the chosen pool for the returned  VIP\footnote{The current implementation of this
application always chooses the first existent pool.}.
Afterwards it updates (third operation) the fetched  pool, along with the newly modified
\texttt{current-member}.
The forth and final operation retrieves
the address information for the selected  Member. 

\note{ discutir a questão de update concorrente (segunda e  terceira operacao )}


\begin{table}[H]
\centering 
\begin{tabular}{l c c c c}
 Operation & Type & Request & Reply \\ \toprule 
 
Get VIP id for the destination IP & R & 104 & 8\\\midrule
Get VIP Info (pool information) & R & 29 & 509\\\midrule
Get the choosen pool & R & 30 & 369\\\midrule
Conditional replace pool after round-robin & W & 772 & 1\\\midrule
Read the chosen Member & R & 32 & 221 \\\bottomrule
\end{tabular}\caption[Workload lbw-0-ip-to-vip( IP packet to a VIP)
operations]{Workload lbw-0-ip-to-vip( IP packet to a VIP) operations
  and sizes (in bytes).}\end{table}

\subsection{Normal Packets}
Even when processing a normal packet, not related to a VIP address at
all, the Load Balancer still has to find out if this is the case. This workload, which only requires one operation (see
table \ref{table:workload:lbw1-1}) sets the minimum amount of work imposed by
the Load Balancer to the controller pipeline. 

\subsection{ARP Request}
This workload  results  from processing an ARP Request addressed at a
VIP address. The data store operations, summarized in Table
\ref{table:workload:lbw1-2}, shows that two reads are
required. First, as previously seen,  it queries the data
store to check if the packet destination address is a VIP (1 read
needed). As it is, the controller then retrieves the \texttt{MAC} address for that
VIP server (so, another read is needed).

\begin{table}[ht]
\small
\centering 
\begin{tabular}{l c c c c}
Operation & Type & Request & Reply \\ \toprule 
Get VIP id for the destination IP  & R & 104 & 8\\\midrule
Get VIP info (proxy MAC address) & R & 29 & 509 \\\bottomrule
\end{tabular}\caption[Workload lbw-0-arp-request( Arp Request to a
VIP) operations]{Workload lbw-0-arp-request( Arp Request to a VIP)
 operations and sizes (in bytes).}
\end{table}



\subsection{Optimizations}

\begin{table}[H]
\small
\begin{tabular}{llccccc}
 Operation & Type &  \multicolumn{5}{c}{ (Request, Reply) } \\  \midrule
&  & lbw-0 & lbw-1  & lbw-2 & lbw-3 & lbw-4 \\ \toprule 
%& &   \multicolumn{5}{c}{(Request, Reply)} \\midrule 
Get VIP id of destination IP  & R & (104,8) &\multirow{2}{*}{(104,509)} &  \multirow{2}{*}{(104,513)} &\multirow{2}{*}{\textbf{(62,324)}} & \multirow{2}{*}{-}    \\\cmidrule{1-2} 
Get VIP info (pool)   & R &  (29,509) & & & &   \\ \midrule 
Get the choosen pool  & R & (30,369)  &  - & (30,373) & -   & \multirow{3}{*}[-2mm]{\textbf{(11,4)}}  \\  \cmidrule{1-2} 
Replace pool after round-robin  & W & (772,1) & -
&\textbf{(403, 1)} &  - \\ \cmidrule{1-2}  
  Read the chosen Member &  R & (32,221) & - & (32,225) & \textbf{(44,4)} & \\\bottomrule  
\end{tabular}\caption[Load Balancer IP to VIP workload operations across
diferent implementations.]{Load Balancer  lbw-\textit{X}-ip-to-vip workload
  operations and respective sizes (in bytes) across diferent
  implementations. Bolded sizes represent significant differences
  across implementations. Sizes marked with \texttt{-} are equal to
  the previous. } 
\end{table}


\begin{figure}[ht]


\begin{floatrow}
\ffigbox{%
  \includegraphics[scale=0.4]{../data/reportGenerator/lbw-0-ip-to-viplbw-1-ip-to-viplbw-2-ip-to-viplbw-3-ip-to-viplbw-4-ip-to-viptxLatCmp.pdf}
}{\caption{Cenas}%
}


\capbtabbox{%
\small
  \begin{tabular}{lll} 
    Prefix &  Data store & Section\\\toprule
    lbw-0 & Simple Key-Value  & \ref{sec:}  \\
    lbw-1 & Cross References  & \ref{sec:} \\
    lbw-2 & Versioned Values & \ref{sec:} \\
    lbw-3 & Column Store & \ref{sec:} \\
    lbw-4 & Micro Components & \ref{sec:} \\ \bottomrule
    & &  \\ 
    & &  \\ 
    & &  \\ 
    & &  \\ 
    & &  \\ 
  \end{tabular}
}{%
  \caption[Name guide to Load Balancer workloads]{Name guide to Load
    Balancer workloads.}\
}
\end{floatrow}
\end{figure}



In the ARP request case  we improve by using cross references tables
to join the two requests into one (lbw-1-arp-request). With the column
based implementation we improve slighty by reducing the size of the
requests as see 

\textbf{Cross References Tables}, So both arp request and the packet address at a VIp workloads show
requests that first obtain the VIP id and only aftewards read the VIP
from the VIP table using the obtained id. So we can improve our
workload footprint by directly obtaining the VIP with only the first
request. 

We can use our cross reference tables mechanism to do that. sec
\ref{sec.datastore.cross.references}. 

\textbf{Versioned Values} With versioned values we can easily  cut the operation which updates
the pool after round robin size in half.  Off course every other get
operation will have an additional 4 bytes on reads. This is negligible  specially in the case of the Load Balancer
since 


\textbf{Columns} 
So with columns lots of changes had to be made.
All value classes (Member, VIP , Pool ) were accessed directly by
their fields.  We \emph{choose} to change that such that are fields
are private and manipulated by getters and setters. 

So we choose to have the Members and VIPS accessed by columns. 
For VIPS we don't actually gain much from this transformation (513 /
324 size reduction)   compared with Members (225 / 4 reduction in
size) . The reason for this is that we actually have to retrieve a lot
of information from the VIP. Since we are caching VIPS and we must
access it in two different types of flow processing (broadcast packet
address to a VIP and packet address to a  VIP) we should on each VIP
request  pull the union of the information required for the two
operations. The reason being is simple. The flow are not
independent. When processing the first we can be sure that unless some
anomaly happens the hosts performing the arp request will consequently
trigger the second flow processing. So we benefit from having the
information in cache to avoid having to perform another request just
because a field is missing. 

Also Members we only require a 4 byte request. 


\textbf{Micro Components} When analyzing the last workload available. We notice two things: we
can't avoid going to the data store for pull and update the round
roubin algorithm. The reason being is twofold: first we should be
sure that the round-robin algorithm is consistent, second even if we
cache the thing then we are in trouble since the update will fail with
the timestamps in order. Off course we use some kind of CRDT value in
the data store which in this case would be pretty simple but then we
would be advocating the extra complexity of the eventual consistency
plane.  So we want to make sure that we got our round robin accurate
but we want to improve on the performance of this workload in
general.  Well, the ideal solution for this would be to use cache for
the VIP as we already have. This a lot of the data store interaction
can be alliviated: we do not have to always consult the data store on
each request since we can cache which address are VIP's and which are
not.  But when an address is a VIP we must consult the data store in
order to round-robin. Ideally we would like to perform all operations
at once in the data store : retrieve the choosen member and update the
pool.  For this we make use of micro-componenets by installing a
prototype method in the data store that does all this for us. 



\subsubsection{Cache}
So we can actually improve on reading the members and vips.  We do not
improve on reading the pools because given the replace operation this
could potentially cause even more operations due to lack of the
consistent pool information. So we transform the application to use
cache on the VIPS and Members  table. On the theoretical best case scenario this
can lower the workload by completely avoiding the first and last
operation. Ideally we should also avoid the normal case of IP packets
not addressed at a VIP. For this our cache  must understand what a
empty value means FIXME. (use containsInCache . update to insert empty
in cache. Then see if containsInCache AND get == null you can be
certain the value is not a VIP), completely avoiding the going to the
data store. 

FIXME: how to measure, justify the time used in the cache parameter? 

\begin{figure}[ht]
\centering
\includegraphics[scale=0.5]{./../data/reportGenerator//lbw-3-ip-to-notviptxLat.pdf}
\caption[Minimum impact of Load Balancer in the pipeline.]{Workload
  lbw-3-ip-to-notvip shows the minimal impact the Load Balancer
  applications has on the pipeline in our best implementation.}
\end{figure}


\section{Device Manager}
\label{sec:feasibility:dm}
\glsresetall
%%%%%%%%%%%%%%%%%%%%%%%%%%%%%%%%%%%%%%%%%%%%%%%%%%%%

The Device Manager application tracks  and stores host device
information such as the switch-ports to where devices are
attached to\footnote{The original application is able to track devices as
  they move around a network, however our current implementation does
  not take this feature into consideration.}. This information ---
that is retrieved from the OpenFlow packets that the controller receives --- is crucial to
Floodlight’s Forwarding application. That is to say, that for  each new flow, the Device
Manager has the job of finding a switch-port for both the destination
and source address. Given this information, it is able to pass it to
the Forwarding application, that can later decide on the actions to
take (e.g., best route). Notice that this arrangement, excludes the
Learning Switch as the  forwarding application in action. 

Regarding the data store usage, Device Manager requires three
tables. The first table, keeps track 
of known devices created by the
application. A second table indexes those same devices by their
\texttt{(MAC,VLAN)} pair.  Finally, a third table maintains an index
from an \texttt{IP} address to one or more devices. Notice
that devices are uniquely identified by their \texttt{id} or \texttt{(MAC,VLAN)} pair, but not their \texttt{IP}\footnote{\textbf{This is true in the context of
  the application. But it does not makes any sense. The only plausible
explanation i see, is that this related with the mobility
tracking capabilities of the original device manager. When considering
mobile hosts, different devices will surely appear with the same
IP.} }.  

We consider two distinct workloads for this application differing in
wether the application already knows the source device information (workload \textbf{dmw1-0})
or not (workload \textbf{dmw1-1}). In the former case, the
application mainly reads information from the data store in order to
obtain location information. As for the latter case, the
application must create the device information and updates all the
existent tables. Therefore, this workload generates more traffic between
the controller and data store. 




A lot of changes had to be made. First to the Device class which is
more than a simple value class and hardly Serializable due to its
static references... 
We lost the Device new DeviceDebugEventLogger() which logged events in
the control plane. 
Avoiding broadcast address querys in the data store. Before they were
free (nearly) not now. 

\begin{figure}
  \centering
  \begin{subfigure}[b]{0.5\textwidth}
                \centering
                \includegraphics[width=\textwidth]{pic/feasibility/dm-unknown}
                \caption{Packet from an unknown device.}
                \label{fig:dm:interaction:unknown}
        \end{subfigure}%
        ~
        \begin{subfigure}[b]{0.5\textwidth}
                \centering
                \includegraphics[width=\textwidth]{pic/feasibility/ls-events-unicast}
                \caption{Packet from a known device.}
                \label{fig:dm:interaction:known}
        \end{subfigure}
        \caption[Device Manager workload events]{Workloads for this application heavily depend on the state of the data store. Unknown devices trigger several operations to the creation of these, while known devices only require an update of their "last seen" timestamp. No matter the case, the source and destination devices are retrieved if they exist.}
        \label{fig:dm:interaction}
\end{figure}


\subsection{Known Devices}
\begin{table}[H]
\centering 
\begin{tabular}{l c c c c}
Operation & Type & Request & Reply \\ \toprule 

Read the source device key & R & 408 & 8\\\midrule
Read the source device & R & 26 & 1444\\\midrule
Update "last seen" timestamp & W & 2942 & 0\\\midrule
Read the destination device key & R & 408 & 8\\\midrule
Read the destination device & R & 26 & 1369 \\\bottomrule
\end{tabular}
\caption[Workload dm-0-known( Known Devices) operations]{Workload dm-0-known( Known Devices) operations and sizes (in bytes).}
\end{table}
When devices are known to the application, a \texttt{packet-in} request
triggers the operations seen in table \ref{table:workload:dmw1-1}. The
first and final operation  read source and destination device
information in order to make their switch-ports available to the
Forwarding process. Additionally, the second operation
(a write), updates the ``last seen'' timestamp of the source
device.  The thorough reader will notice that the request size of this
operation significantly exceeds the size of  a timestamp. This is
taken in consideration when we optimize this application (see
section TODO). Right now it suffices to say that the lack of a more
sophisticated data model  and appropriate concurrency control in the
data store interface leads to excessive information exchange. 

\subsection{Unknown Source}
\begin{table}[H]
\centering 
\begin{tabular}{l c c c c}

 Operation & Type & Request & Reply \\ \toprule 

 Read the source device key & R & 408 & 0\\\midrule
Get and increment the device id counter & W & 21 & 4\\\midrule
Put new device in device table & W & 1395 & 1\\\midrule
Put new device in \texttt{(MAC,VLAN)} table & W & 416 & 0\\\midrule
Get devices with source IP & R & 386 & 0\\\midrule
Update devices with source IP & W & 517 & 0\\\midrule
Read the destination device key & R & 408 & 8\\\midrule
Read the destination device & R & 26 & 1378 \\\bottomrule
\end{tabular}
\caption[Workload dm-0-unknown( ARP from Unknown Source)
operations]{Workload dm-0-unknown( ARP from Unknown Source) operations
  and sizes (in bytes).}
\end{table}


This workload is triggered in the specific case in which  the source device
is unknown and the OpenFlow message carries an ARP request
packet. Seing that both these  conditions are true, the application
proceeds  with 7 data store operations, described in table
\ref{table:workloads:dmw1-1}. Their intention is to  create device
information and update the three tables described  in the beginning
of this section.  The first operation reads the  source device. Being
that it  is not known, this operation fails (notice in the table, that
the reply has a size  of zero bytes). As a result the application
proceeds with the creation of a device. For this, the
following write (second operation) atomically retrieves
 and increments a device
unique \texttt{id} counter. Afterwards, the third and fourth  operation
update, with the newly created device, the device and MAC/VLAN
tables respectively. Likewise, the fifth and sixth operations update
the \texttt{IP} index table. Given that this index links an \texttt{IP} to
several devices we are forced to first collect the set of devices in
order to update it\footnote{Again, this can be alleviated with a more
  expressive data model.}. This \emph{read-modify} operation can
fail in case of concurrent updates. Under that case, both operations
would be repeated until they succeed. At this point, the Device Manager
is done with the creation of the device and can, finally, move to the
last operation which reads the destination device information. 

\subsection{Optimizations}
\begin{figure}
  \centering
  \begin{subfigure}[b]{0.5\textwidth}
                \centering
                \includegraphics[width=\textwidth]{../data/reportGenerator/dm-0-unknowndm-1-unknowndm-2-unknowndm-3-unknowndm-4-unknowntxLatCmp.pdf}
                \caption{}
                \label{}
        \end{subfigure}%
        ~
        \begin{subfigure}[b]{0.5\textwidth}
                \centering
                \includegraphics[width=\textwidth]{../data/reportGenerator/dm-0-knowndm-1-knowndm-2-knowndm-3-knowndm-4-knowntxLatCmp.pdf}
                \caption{}
                \label{}
        \end{subfigure}
        \caption[Device Manager performance analysis]{}
        \label{fig:dm:performance}
\end{figure}


\begin{table}
\small
\begin{tabular}{lll} 
    Prefix &  Data store & Section\\\toprule
    dm-0 & Simple Key-Value  & \ref{sec:}  \\
    dm-1 & Cross References  & \ref{sec:} \\
    dm-2 & Versioned Values & \ref{sec:} \\
    dm-3 & Column Store & \ref{sec:} \\
    dm-4 & Micro Components & \ref{sec:} \\ \bottomrule
  \end{tabular}
  \caption[Name guide to Device Manager workloads]{Name guide to
    Device Manager workloads.}
  \label{table:names:dm}
\end{table}



\begin{table}[H]
\small
\centering
\begin{threeparttable}
\begin{tabular}{ll ccccc}
 Operation & Type &  \multicolumn{5}{c}{ (Request, Reply) } \\  \midrule
&  & lbw-0 & lbw-1  & lbw-2 & lbw-3 & lbw-4 \\ \toprule 
Get source key & R &(408, 8) & \multirow{2}{*}{(408,1274)} &
\multirow{2}{*}{(408,1278)} & \multirow{2}{*}{(486,1261)} &
\multirow{2}{*}{(28,1414)} \tnote{a} \\ \cmidrule{1-2}
Get source device & R & (26,1444) & & & & \\ \midrule
Update timestamp & W & (2942,0) & (2602,0) & (1316,1) & (667,1) & 
(36,0) \\ \cmidrule{1-2}
Get destination key & R & (408,8) & \multirow{2}{*}[-1mm]{(408,1199)} &
\multirow{2}{*}[-1mm]{(408,1203)} & \multirow{2}{*}[-1mm]{(416,474)} &
\multirow{2}{*}[-1mm]{N/A} \\ \cmidrule{1-2}
Get destination device & R & (26,1369) &  &
 & & \\\bottomrule
\end{tabular}
\caption[Workload dm-0-known( Known Devices) operations]{Workload
  dm-0-known( Known Devices) operations and sizes (in bytes).}
\begin{tablenotes}
\item [a)] This operation also fetches the destination device.
\end{tablenotes}
\end{threeparttable}
\end{table}

%TODO - do not use put new device in MAC,VLAN table. This is
%confusing. 

\begin{table}[ht]
\small
\centering 
\begin{threeparttable}
\begin{tabular}{ll ccccc}
 Operation & Type &  \multicolumn{5}{c}{ (Request, Reply) } \\  \midrule
&  & lbw-0 & lbw-1  & lbw-2 & lbw-3 & lbw-4 \\ \toprule 
Read source key & R & (408,0) & - & - & (486,0) & (28,201)\tnote{a}\\
Increment counter & W & (21,4) & -  & - & - & \multirow{5}{*}{(476,8)} \\
Update device table & W & (1395,1) & (1225,1)\tnote{b}  & - &
(1183,1) & \\
Update MAC  table & W & (416,0) & - & - & -
& \\
Get from IP index & R & (386,0) & - & - & - & \\
Update IP index  & W & (517,0) & - & - & - & \\
Get destination key & R & (408,8) &
\multirow{2}{*}{(408,1208)}\tnote{b} & \multirow{2}{*}{(408,1212)} &
\multirow{2}{*}{(416,474)} & \multirow{2}{*}{N/A}  \\ 
Get destination device & R & (26,1378)  &  & & \\\bottomrule
\end{tabular}
\caption[Workload dm-0-unknown( ARP from Unknown Source)
operations]{Workload dm-0-unknown( ARP from Unknown Source) operations
  and sizes (in bytes).}
\begin{tablenotes}
\item [a)] This operation also fetches the destination device.
\item [b)] Differences in sizes caused by a SERIALIZATION improvement 
\end{tablenotes}
\end{threeparttable}
\end{table}

%the question that may arise: why not simply put everything inside the
%data store then? Well, in the case of the device manager, there are
%actually a few subitilidades hidden in our explanation. First there
%is a dependency of others services. topology manager. 



\subsubsection{Cross References tables}
So in practice we have to get the device key from the mac/vlan table and only then we can actually get the device information we want. This is redudant, and we can actually perform the all operation instantenously  by using cross references tables.   This reduces the
number of messages. In this workload we also tuned the device n
serialization process to lower the devices message size.  We simple
remove the reference to the entityClassifier since classifiers are
known to the local controller and we actually only need the
entityClassifier name. Its instance/class can then be obtained. 




\subsubsection{Timestamps}
\subsubsection{Cache}
With cache we fetch known devices in an optimistic concurrency
manner.  If there is no such device in cache, we then try to obtain it
from the data store, as it might be created from other
controller. (Really? - yes same device with different routes that go
through different openflow controllers)

What we try to do is to fetch device from the cache. At some point in
time in a  normal network, we hope that all device information is
known. After that point devices in cache should pass the timestamp
update at first (if they are not updated by concurrent
controllers). If the devices are connected to different openflow
island simultaneously than this is a bad idea since we will actually
have to perform one more request that the normal workload
pattern. (try to updated - fails, retrieve new , update) . Off course
this could be mitigated by having the update attempt to return the
currently present value timestamp.. 


\subsubsection{Columns}
Biggests changes. Joshua Blosch advice. To be fair, we have also
broken this rule every now and then. 

Other applications can add and manipulate what kind of columns they
want for the devices. For example in our scenario only the attachment
points of the destination device are needed. But other applications
could require the complete list of MAc addresses known for that
device...  They could easily do that by manipulate the DeviceManager
interface to add this requirement to the list of fetched information
from the Device. 

\subsubsection{Micro Components}
Just a proof of concept. Establish special methods in the data store. 
Three : 
createDevice <-
updateDeviceTimestamp <- 
getTwoDevices <- could be done with transactions.  Gets the source
device entirely. Also gets the attachments points of the destination
device since these are required to forwarding. 

%%% Local Variables: 
%%% mode: latex
%%% TeX-master: "../PEI"
%%% End: 

\message{ !name(../PEI.tex) !offset(-680) }

\end{document}
