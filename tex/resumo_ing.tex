% ----------------------------------------------------------------------
% P�gina do resumo em Ingl�s:
\selectlanguage{english}
\vspace*{2cm}
\begin{center}
\Large \bf Abstract
\end{center}
\vspace*{1cm} \setlength{\baselineskip}{0.6cm}

This work is motivated by the 
emergent network architecture of Software
Defined Networks where the control of the network is extracted from the
network devices and delegated to a
server named controller that is responsible for dynamically
configuring the network devices present in the infrastructure. The
controller has the advantage of logically centralizing the network
state in contrast to the previous model where state was distributed
across the network devices. If spite of this logical centralization,
the control plane (where the controller operates)  must be distributed
in order to support the scale of modern datacenters and
networks. However, this distribution introduces several challenges due
to the heterogeneous, asynchronous and faulty environment where the
controller operates. Current distributed controllers
lack transparency due to the  eventual consistency
properties employed in the distribution of the controller. This results in a
complex programming model for the development of network control applications. This work will
contribute with a fault-tolerant distributed controller with strong
consistency properties that allows a more transparent distribution of
the control plane. Also we believe that even if favoring 
consistency, we will be able to provide superior performance results
that those available in the literature. 


% The success of the current Internet infrastructure and architecture is
% indisputable. However, it  has long suffered from negative
% criticism. The "classic'' Internet architecture suffers from serious
% drawbacks in the management field as the inherent complexity
% associated with the manual configuration of  distributed 
% network equipment can be too much to bare with only rudimentary
% mechanisms as command line interfaces and hand-made configuration
% scripts. Human error is known to be the most probable cause of network
% downtime. Additionally the lack of control abstractions  in
% networking  prevents the deployment of both new and modified control functions.

% These problems are cumulative as both network designers
% and administrators struggle to satisfy modern network
% requirements. Currently several control mechanisms are designed,
% deployed and configured in isolation of each other. Known examples
% are:intra domain routing; inter domain routing; access control; QoS services; packet inspection;load-balancing; and intrusion detection. These control mechanisms
% are deployed in network devices such as routers and switches with no
% support for their modification. 
% Thus, devices that should  be optimized for data forwarding end up bundled with
% heavyweight control features that must be configured independently. 

% Software Defined Networking (SDN) radically changes the current IP
% architecture by extracting the control function from the network
% devices and shifting it to a centralized service known as the control
% plane. This plane, with the help of a coherent view of the network state
% can dynamically configure the network devices.
% The network state
% is centralized in a datastore known as Network Information Base (NIB)
% or View. The NIB  can provide up
% to date connectivity, namespace information, and other physical or
% logical information capable of satisfying network control  objectives. 
%  The result configuration is the output
% of a function applied to the current network state. In SDNs networks devices are
% only responsible  for packet forwarding  as opposed  to the classic
% Internet architecture where devices also run distributed algorithms
% in order to perform self-configuration. 

% Centralized control planes have been reported to scale to tens of
% thousands of hosts. However the current
% state of the art implementations are still far from satisfying the
% requirements of modern datacenters and Wide Area Networks as the high traffic present in these networks may lead a
% centralized service to exhaustion. In addition latency issues presented
% in  WAN's covered by a centralized control planes may not
% be acceptable under the quality of service desired. Finally a single
% point of failure is not an usual option  as the failure of the control plane
% may compromise network availability. All this  strongly
% suggests the distribution of the control plane. However, the
% distribution processes must not affect the benefits of the
% SDN architecture even if operating in a heterogeneous, asynchronous and faulty
% environment.

% Although SDN has been considered a hot
% research topic the existing work does not covers distribution of the
% control plane extensively. Two designs are often-cited: HyperFlow \cite{Tootoonchian:2010vy} and
% Onix \cite{Koponen:2010th}. The latter replicates network events that cause changes to the NIB
% while the former replicates and distributes the NIB state. We consider both those works to have serious
% drawbacks as lack of distribution transparency and weak consistency
% properties leading  to a complex programming model for
% the development of network control applications. 

% In our work we plan  to contribute to the Software
% Defined Network field by introducing a strong consistent Network
% Information Base integrated in an open source SDN controller. Our NIB
% is replicated across controllers instances in a distributed state machine
% fashion that we implement with  well-known protocols to the  Distributed
% Systems field. We believe that the Network community has overlooked
% both the correctness and simplicity advantages of strong consistent datastores as
% well as their performance. We
% also belief that correctness problems can arise from the use of 
% weak consistency datastores if the application is not fully aware of
% the distribution guarantees given by the NIB. 


%Resumo at� \textbf{300} palavras. 

\vfill

\begin{flushleft}
\textbf{Keywords:}
Replication, Strong Consistency, Distributed State Machine,
Distributed Control Plane, Software Defined Networking. 
\end{flushleft}

\LIMPA
\selectlanguage{portuguese}
% Fim da p�gina do resumo em Ingl�s.
% ----------------------------------------------------------------------
