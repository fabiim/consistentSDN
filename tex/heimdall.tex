

\section{Controller}
The most interesting thing about this design is that each controller instance can have direct control over the window of inconsistency for every application. 
\subsection{Architecture}

\begin{figure}[ht]
  \centering
  \begin{subfigure}[b]{0.4\textwidth}
                \centering
                \includegraphics[width=\textwidth]{pic/heimdall/global-app}
                \caption{General architecture}
        \end{subfigure}%

        \begin{subfigure}[b]{0.7\textwidth}
                \centering
                \includegraphics[width=\textwidth]{pic/heimdall/instance-design}
                \caption{Controller instance architecture}
        \end{subfigure}
\end{figure}



\begin{description}
\item[Global Application] A global app can be a monitoring app, a web based interface to update network policies, etc., Those apps do not fit in the event-driven programming  model of the controller.  As such they do not need to be inside each controller but rather on a global location that has access to a controller instance. Each controller instance has a REST API that exposes network and data store state according to the native apps decisions and configurations;
\item[Native Application] A native app resides in the controller memory and is an event based application that usually reacts to network changes, flow requests and data store changes. These applications can be replicated across each controller instance. Examples are the Topology Manager, Device Manager, Firewall, Load Balancer and Forwarding applications; 
\end{description} 

\noindent\makebox[\textwidth]{\rule{\paperwidth}{0.4pt}}

Controller Core : 

\begin{description}
  \item[Pipeline Manager] Manages the dispatching of  events to the applications. Applications are registered in this Manager to handle specific events. Events can be triggered by the data plane or the applications itself. When an event  occurs the dispatcher calls applications in a predefined order;
  \item[Pooling Manager] Applications register routine timed operations. Ideally this will support efficient pooling to check changes and/or updates in both the data store and data plane (e.g., counters); 
  \item[Data Store Service] Allows the creation and/or location of tables in the data store. Clients use this to create local objects to manipulate data store tables. Also manages the Cache inside the controller; 
  \item[Switch Communication] Communication with switches (e.g., install rules, consult flow tables). Applications will not have the possibility to directly contact the data plane. This module abstracts this into a service that can later be expanded to handle more complex tasks such as conflicting policies, safety checking, table space reduction etc.,; 
  \item[Network Partition] Handles the controller instance reliability protocol (take over other controllers that have failed);
\end{description}



\subsection{Transactional Memory}

This section presents a rude idea to approach the controller pipeline (processing events by different applications) focusing in performance. The problem it tries to solve is the excessive communication with the data store. 
It is inspired by the micro components approach (having code in the data store such as stored procedures in SQL data bases), transactions and transactional memory. The basic idea is to avoid the data store at all cost. When communicating with the data store, join requests from different switches and applications to batch them. That is it. It may not work every time, but it is still possible to resort to the typical approach in isolated cases.\\


The pipeline: 
\begin{itemize}
\item The controller pipeline is where each application is registered to process the event in order. 
\item A pipeline is defined by the event type. For each event there is a  pipeline with different application as stages. 
\item Each pipeline will be run by a single thread  defined by the source of the event (usually a switch). This way there is no need for locks and concurrency control since each event source will have a private state (e.g., cache). The pipeline is lock-free. 
\end{itemize}

Figure \ref{fig:pipeline} shows the general behaviour of the pipeline. 


\begin{figure}[ht]
  \centering
  \includegraphics[width=0.9\textwidth]{./pic/heimdall/pipeline.pdf}
  \caption{Processing events in the pipeline. A single thread pre-fetches everything that is not existent in cache to the data store. Then each app processes the event locally with no data store communication. In the end we commit to the data store the collection of modifications done by every application. If sucessfull we can proceed to installing a flow rule if necessary. }
\label{fig:pipeline}
\end{figure}


\begin{itemize}
\item The first step is to call each application handler that receives the event and  tells which data it needs to fetch to cache . 
\item The data is fetched to cache in a single operation that groups all the requests from all applications. 
\item Then, we proceed with the event processing by calling each application handler. This time applications do not perform any communication with the data store. Everything is done locally. 
\item In the end we attempt to commit every change made by each application in the data store with a single operation. Notice that this commit can fail if the user wishes to only succeed if the data fetched in the cache phase as not been modified in the data store when we are committing. This enables consistency control of stale data (cache based) if desired. 
\item Finally we can build the outcome of processing the event (e.g, install rules in the switches). 
\end{itemize}

Notes : 
\begin{itemize}
\item We may need an excessive amount of micro components to group operations that need to read, process and then read again. Ideally you want two micro components ``per event'' : one to read stuff into cache, and one to write. When the write happens every time you might as well have a single micro component which does everything in the data store; 
\item Some data may already be present in the cache. So we can avoid reading from the data store in the pre-fetch phase;
\item Some events may not need to write anything to the data store;
\item We could consider Pooling for writes (periodically  update the data store). Imagine: you do not care if a device timestamp is updated 10/20 times in a minute. Every 30 seconds is fine.  
\end{itemize}

Figure \ref{fig:pipeline-seq} shows a possible sequence diagram when processing an event in the pipeline. 

\begin{figure}[ht]
  \centering
  \includegraphics[width=\textwidth]{./pic/heimdall/pipeline-seq.pdf}
  \caption{Processing events in the pipeline. }
  \label{fig:pipeline-seq}
\end{figure}


\subsection{Data Store}

\subsubsection{Domains}
\begin{itemize}
\item A data store domain is an instance of the bft-smart system with a given number of replicas;
\item With different domains we can increase the throughput and latency of the system ; 
\item We can shard tables over domains. For example we could have a global domain for all global tables (accessible by different controllers). Local tables could be available in another domain. At the extreme each controller would have its private domain and a global one. 
\end{itemize}

\subsubsection{Data Store Service}

\begin{itemize}
\item Applications only communicate with each other through the data store. 
\item What application is responsible for some data is not relevant
\item The applications known through the data store service if some kind of application (and respective tables is available) 
\end{itemize}


The data store supports different table types shown in Table~\ref{tab:table-types}. The Global and Global to Application tables must be made available to every controller while the Local and Local to Application can be reachable only to a single controller instance. However in case of failures and controller fail-over one must be certain the backup controllers can reach the failed controller tables. Beyond the types shown each table state can be persistent (kept in disk) or not. The former is of the utmost importance for the controller configuration data (e.g., network policy). Notice however that the state is always kept in memory. If the controller state grows beyond the memory size  of a data store replica then the data can be divided across different domains at the expense of more hardware. 

\begin{table}[ht]
  \centering
  \begin{tabular}{ll}
    Name & Accessibility \\ \toprule 
    Global & Every controller and every application   \\ 
    Global to Application & Single application on every controller \\
    Local & Every app in a single controller \\
    Local to Application &  Single application in a single controller \\ \bottomrule 
  \end{tabular}
  \caption{Table types}
  \label{tab:table-types}
\end{table}


\begin{figure}[ht]
  \centering
  \includegraphics[width=\textwidth]{./pic/heimdall/global-to-local.pdf}  
  \caption{Topology tables in the data store. The local topology represents the entire topology seen by a controller. The global topology shows only the minimum spanning tree.}
  \label{fig:topologies}
\end{figure}

Performing aggregation: 

\begin{itemize}
\item Local Tables can have code in the data store; 
\item Under each write in the local table the code checks if something needs to be written to the global table; 
\item For example Figure~\ref{fig:topologies} shows that the local version of a controller topology table shows every link while the global version (available to every controller) only shows a minimum spanning tree of the local topology.  A controller could even choose to hide the entire topology and just keep all its devices as attached to the border switches (that connect to other controllers data plane domains). 
\end{itemize}

The Data Store Service (DSS) as viewed from the user : 
\begin{itemize}
\item Allows the creation of local objects to communicate with the data store. 
\item Manages the creation of the controller cache, and transactional pipeline entities (Cache Manager, Write Manager). 
\item Guarantees that each event thread gets an isolated environment (its own cache and data store channel). This is important to make the pipeline lock free and also to improve concurrency (different data store channels). 
\item Shields the user from the RPC semantics in the communication with the data store. 
\item Provides a static typed interface for the data store. 
\item Expanded in the future with some kind of query language, which is essential to applications. This will enable a more expressive communication with the data store that is able to filter the values obtained. 
\end{itemize}
\
\subsection{Pooling}
\begin{itemize}
\item Typically controllers are reactive. Floodlight applications have a publish/subscribe behaviour to warn other applications of events. An example is the Device Manager which triggers events for devices that have been created, moved or deleted. The Topology Manager does the same thing on every topology change. 
\item We do not want this behaviour. This would require sending information to all controllers (which would not make sense to do trough the data store). 
\item So each application maintains (if she wishes) a pooling task in the pooling manager. 
\item The pooling manager groups tasks by subscriptions and periodicity. 
\item Checking changes is done with version numbers by table, table entry and (possibly) columns. The data store should have the possibility of tracking changes to the tables between versions. This is crucial for efficiently fetching the devices, or links that have been added or removed between versions. 
\item When changes are made, a pipeline is formed for the applications that subscribed to those changes. 
\end{itemize}

\section{Reliability}
A simple ``protocol'' to handle switch fail-over. I still need to think of bad things that can happen when two controllers fight over a switch.
\subsection{Protocol}
\begin{itemize}
\item There is an eventual failure detector built with the data store. If the controller does not has access to the data store it should not control anything in the network (Fig.~\ref{fig:proto-heartbeats}). The data store also keeps the switches under each controller ; 
\item Controllers can suspect the failure of another and attempt to replace it. First they have to set a flag in the data store stating that they are attempting to take over the presumably failed controller (Fig.~\ref{fig:proto-c2-fails}). With this flag other controllers can known that someone is attempting to control the switches and avoid taking over. This also helpfully for the failed controller which can recover and known that there is an active process to take its place. 
\item A controller can then set himself as master of the switch. After this point all switches messages go to the new controller (C1) (Fig.~\ref{fig:proto-c1-as-master}); 
\item Finally the controller sets himself as master of the switch in the data store. (Fig.~\ref{fig:proto-c1-finishes}). 
\end{itemize}


\begin{figure}
  \centering
  \begin{subfigure}[b]{0.5\textwidth}
                \centering
                \includegraphics[scale = 0.4]{./pic/heimdall/proto-1}
                \caption{Heartbeats sent to the data store. }
                \label{fig:proto-heartbeats}
        \end{subfigure}%
        ~
        \begin{subfigure}[b]{0.5\textwidth}
                \centering
                \includegraphics[scale=0.4]{./pic/heimdall/proto-2}
                \caption{C1 finds out that C2 has failed.}
                \label{fig:proto-c2-fails}
        \end{subfigure}

  \begin{subfigure}[b]{0.5\textwidth}
                \centering
                \includegraphics[scale = 0.4]{./pic/heimdall/proto-3}
                \caption{C1 has to set himself as master in S2. }
                \label{fig:proto-c1-as-master}
        \end{subfigure}%
        ~
        \begin{subfigure}[b]{0.5\textwidth}
                \centering
                \includegraphics[scale=0.4]{./pic/heimdall/proto-4}
                \caption{Finally tell the data store he controls S2. }
                \label{fig:proto-c1-finishes}
        \end{subfigure}
\end{figure}

Notes
\begin{itemize}
\item In the case S1 fails while taking over S2, other controllers (if any) will eventually known about this and take over. 
\item This process can also be triggered in the case a switch fails. It is worth doing so because a switch can fail due to network failures who affect the connectivity with his master controller but not with others; 
\item As long as one  controller does not fail and reaches every switch (who also has not failed)  and data store we can handle all kind of network failures.
\item When C2 recovers he can do the same process to take over his switches again. 
\item I think that to be safe (I will try to define safety later) the data store should deny state changing messages from controllers who do not own the switch (which caused the state changes). But I have to think better about this.... 
\end{itemize}

% \linebreak

% \section{Shared Data Store Controller Architecture}
% \glsresetall
% \label{sec:heimdall:architecture}
% %%%%%%%%%%%%%%%%%%%%%%%%%%%%%%%%%%%%%%%%%%%%%%%%%%%%%%%%%%%%%%%%%%%%%%%%%%%%%%%%%%%%%%%%%%%%%%%%%%%%%

% The proposed distributed control architecture is based on a set of controllers acting as clients of the fault-tolerant replicated key-value data store, reading and updating the required state as the control application demands, maintaining thus only soft state locally.
% There are two main concerns around this design: (i) how to avoid the storage being a single point of failure and (ii) how to avoid making the storage a bottleneck for the system.
% In the previous section we showed that state-of-the-art state machine replication can be used to build a data store that solves both these concerns.


% Fig. \ref{fig:architecture} shows the architecture of our shared data store distributed controller.
% The architecture comprises a set of SDN controllers connected to the switches in the network.
% All decisions of the control plane applications running on the distributed controller are based on OpenFlow events triggered by the switches and the consistent network state the controllers share on the data store.
% The fact that we have a consistent data store makes the interaction between controllers as simple as reading and writing on the shared data store: there is no need for code that deals with conflict resolution or the complexities due to possible corner cases arising from weak consistency.

% By design, the SMR-based data store is replicated and fault-tolerant (as in all designs discussed in the previous section), being up and running as long as a majority of replicas is alive~\cite{Lam98}.
% In other words, $2f+1$ replicas are needed to tolerate $f$ simultaneous crashes.
% Thus, besides offering strong consistency, this architecture leads to a completely fault-tolerant control plane.
% Furthermore, in this design the controllers keep only soft state locally, which can be easily reconstructed after a crash.
% The switches tolerate controller crashes using the master-slave configuration introduced in OpenFlow 1.2\,\cite{ONF2011}, which allows each switch to report events to up to $f+1$ controllers (being $f$ an upper bound on the number of faults tolerated), with a single one being master for each particular switch.
% The master is constantly being monitored by the remaining $f$ controllers, which can takeover its role in case of a crash.

% Interestingly, our architecture could also be used in SDN deployments were a distributed controller is not necessary, to implement fault tolerance for centralized controllers.
% In this case the fault-tolerant data store can be used to store the pertinent controller state, making it extremely simple to recover from its crash.
% In this case, the applications deployed on the primary controller manage the network while a set of $f$ backup controllers keep monitoring this primary, just as in the distributed controller design.
% If the primary fails, one of the backups -- say, the one with the highest IP address -- takes the role of primary and uses the data store to continue controlling the network.

% Our distributed controller architecture covers the two most complex fault domains in an SDN, as introduced in~\cite{kim2012}.
% It has the potential to tolerate faults in the controller (if the controller itself or associated machinery fails) by having the state stored in the fault-tolerant data store.
% It can also deal with faults in the control plane (the connection controller-switch) by having each switch connected to several controllers (which is ongoing work).
% The third SDN fault domain --- the data plane --- is orthogonal to this work since it depends on the topology of the network and how control applications react to faults.
% This problem is being addressed in other recent efforts~\cite{kim2012,Reitblatt2013}.

% \begin{figure}
% \centering
% \includegraphics[scale=0.6]{./pic/heimdall/multicontroller.pdf}
% %add desired spacing between images, e. g. ~, \quad, \qquad etc.
% %(or a blank line to force the subfigure onto a new line) 
% \caption[Heimdall Architecture]{The shared data store controller
%   architecture with each switch sending OpenFlow messages to two
%   controllers. The controllers coordinate their actions using a
%   logically centralized data store, implemented as a set of
%   synchronized replicas. }
% \label{fig:architecture} 
% \end{figure}

% \section{Floodlight} 
% \glsresetall
% \label{sec:heimdall:floodlight}
% %%%%%%%%%%%%%%%%%%%%%%%%%%%%%%%%%%%%%%%%%%%%%%%%%%%%%%%%%%%%%%%%%%%%%%%%%%%%%%%%%%%

% \section{Data Store}
% \glsresetall
% \label{sec:heimdall:dataStore}
% %Features  design, etc., 
% \label{sec:heimdall:datastore:bft-smart}
% FIXME : Alysson - será que posso justificar que a data store
% performance nao é importante quando comparada com o middleware? 

% \subsection{Smart}
% we had to change the library used of netty from netty-3.1.1.GA.jar to
% netty-3.2.6.Final.jar because it conflicted with floodlight. We are in
% the dark here. Do not known if this will cause problems... 

% \subsection{Implementation}

% \subsubsection{Map interface} 
% Actually motivated by the LearningSwitch application. The rest interface exposed the learning switch database (hash table ) directly. We implemented a class to Delegate the map interface methods to our KeyValueTable. We don't actually use it inside the applications we modified because we have some preference for static typing (which for arguably legitimate reasons does not appear in the Map interface). Nevertheless we could use it. putAll and containsValue are not implemented. No special reason just lazinesss.  

% LinkedHashMap should be used for maintaining consistent ordering
% across replicas. 
% \pagebreak 
% \section{Data store}
% \label{sec:heimdall:datastore:functionalities}

\subsection{Prototype Implementation}
\textbf{Isto vai ser uma secção a falar do protótipo de base de dados implementado.} 
\begin{figure}[ht]
  \centering
  \includegraphics[scale=0.6]{./pic/heimdall/client-tables-classes.pdf}
  \caption{Class diagram of the client interface to the data store tables.}
\label{fig:design:class-diagram}
\end{figure}

Missing things to talk about: 
\begin{itemize}
\item An hash table and a Key Value table is the same, we use both terms. 
\item Define \gls{rpc} messages as the messages that are sent to the data store to perform an operation. 
\item Define marshalling/unmarshalling as the process to convert an object into a byte representation and vice-versa. 
\end{itemize}

We implemented a  data store prototype that has been iteratively refined to incorporate data store functions required to increase the performance of real world applications (discussed in Chapter~\ref{sec:feasibility:apps}).  
Fig~\ref{fig:design:class-diagram} shows the class diagram for the client side interfaces that allow him to communicate with the  data store.  
\begin{itemize}
\item \emph{ITable} interface introduces the general functionality of a key value table.
\item \emph{IKeyValueTable}  is a normal hash table. You can manipulate the key to value association in different ways. It extends a \emph{ITable}
\item \emph{IColumnTable} is the extension of a \emph{IKeyValueTable} into a bi-dimensional table where two keys access an individual value. 
\item \emph{ICachedKeyValueTable} allows explicit control over the window of inconsistency accepted in cached values for an \emph{IKeyValueTable} 
\item \emph{ICachedColumnTable} does the same for an \emph{IColumnTable} 
\end{itemize}

This is not a lesson in design. The interfaces presented are part of a prototype that we leveraged to capture the behaviour of applications, and perform a requirement study for the data store. Off course that in production code we can use any of the existent off the shelves data stores (either SQL or NoSql) given that they offer the same functionalities that we present in this section. 
\label{sec:heimdall:key-value}

\subsection{Cross References} 
\label{sec:heimdall:cross-references}
A Cross Reference table ($t_{cr}$) contains values that can be used as keys in another table ($t_{d}$). 
Therefore a data store client can use a key valid in  one table ($t_{cr}$) to retrieve a value at  a different table ($t_{d}$) but with the benefit of using a  single data store operation (the \texttt{getCrossReference} method at \emph{IKeyValueTable}). 

Commonly, despite the number of unique attributes that can be used to identify a value, we are limited to using a single one when using an hash table. 
Commonly, an hash table is restricted to a single key to identify a value despite the  number of unique attributes  that can be used to identify it. 
Furthermore, the asymptotical complexity to obtain a value with a particular key is \BigO{1} as opposed to searching for one which, at best, has $\Omega(\log n)$ complexity for $n$ entries (using balanced trees). 

To circumvent those limitations, one can use an additional table that relates a ``secondary'' key of a value to its ``main'' one. 
To clarify, imagine that for the purpose of tracking hosts in a network we consider that a device is uniquely identified by an \gls{ip} or \gls{mac} address. 
Therefore, we could use two tables: one relating \glsplural{ip} to  \glsplural{mac}, and another relating \glsplural{mac} with devices.

This is a reasonable scheme in a local environment (in memory hash table) given that the asymptotical cost to obtain a device with 
a \gls{mac} address or its \gls{ip}  are equal (\BigO{1}). 
However, in a distributed environment, this scheme requires two round trips to the data store just to obtain a single device with a \gls{ip} address (one to fetch the \gls{mac}, and another to fetch the device). 

Seing that this was a common behaviour in the existent Floodlight applications that we modified to use our data store, we developed the Cross Reference functionality which, as shown in chapter~\ref{sec:feasibility:apps}, results in a significant performance improvement. 
With it, the clients are able to create multiple Cross References tables to track values using different uniques attributes. 
As a result, despite the key that a client uses to obtain a value it can always do it with a single operation. 


\subsection {Versioning}
\label{sec:heimdall:versioning}
\begin{figure}[ht]
  \centering
  \begin{subfigure}[b]{0.5\textwidth}
                \centering
                \includegraphics[width=\textwidth]{./pic/heimdall/versioning-0}
                \caption{No Versioning.} 
                \label{fig:heimdall:versioning-0}

        \end{subfigure}
        ~
        \begin{subfigure}[b]{0.5\textwidth}
                \centering
                \includegraphics[width=\textwidth]{./pic/heimdall/versioning-1}
                \caption{Versioning.}
                \label{fig:heimdall:versioning-1}
        \end{subfigure}
  \caption[Concurrent updates]{The concurrent update to the \texttt{visitors} set for a particular site can result in loss of data. In  Fig.~\ref{fig:heimdall:versioning-0} the update from client 1 is forgotten  when replaced by the update from client 3 (last-write-wins).  Conversely in Fig~\ref{fig:heimdall:versioning-1} the use of versioning in the data store prevents client 2 from overwriting the last update.}
\label{fig:heimdall:versioning}
\end{figure}

With Versioning we associate each table entry (i.e., key value pair)  with a monotonically increasing counter --- the version number ---   that is incremented in every mutation operation. 
Doing so, we empower the data store with the capability to detect and prevent conflicting updates that otherwise could result in the loss of data. 

To clarify, imagine an \gls{http}\footnote{The application protocol for distributed, collaborative, hypermedia information systems.} network logger running in a controller that maintains  a Key-Value table  (in the data store) to map each \gls{url}\footnote{A uniform resource locator, also known as web address,  that constitutes a reference to a resource such as a web page or email.} seen to the set of \gls{ip} addresses that have visited it. 
Whenever a hosts visits a site, the controller adds the \gls{ip} address of that host to the site visitors set. 
However, being that  for the data store a set is merely an opaque binary object, the controllers are forced to fetch the set, add an element locally and finally write the new set in the data store. If two controllers do this concurrently, then it is possible to loose values added to the set. 

Fig.~\ref{fig:heimdall:versioning-0} shows how this update algorithm can result in data loss in the context of concurrent updates. 
First, controllers 1 and 2 fetch the same \texttt{visitors} set for a particular site (uniquely identified by the \gls{url}), then they replace it by a new set that includes IP2 and IP3 respectively. 
In this case the lack of concurrency control  results in the loss of the write operation that includes the IP1 visit to the site (visitors={IP2,IP3})  since the  last write (visitors = {IP1, IP3}) overwrites the previous. 

Fig~\ref{fig:heimdall:versioning-1} shows that with Versioning, the write from controller 1 results in an increase of the version number of the visitors set at the data store (to 2) which prevents any update done by controllers unaware of the most recent version of the set. 
Therefore, by the time the second write request (from controller 2) arrives at the data store it can be aborted since  the version number included in the operation is not consistent with the data store. 


Whenever the data store denies a request (as in the example above) the client can only repeat the entire process since that in order to complete its write, it must obtain the current version number which is obtained from reading the value from the data store. 
To be true, it is often the case that the correct behaviour requires reading the value again, since the client write is built over the existent value in the data store. 
Otherwise, the client would have used a ``free'' write operation not subjected to the data store version verdict. 


It is important to realize that the data store has no mechanism to guarantee that a stubborn client will eventually succeed.
Indeed, it is possible that one client  loops indefinitely if another client constantly out-wins him in every write attempt. 
Clients are solely accountable for  guaranteeing  the progress (liveness) of their updates. 
This process is commonly termed of \emph{Optimistic Concurrency Control}. Clients are optimistic in the sense that they hope that no one else updates the value while they perform the entire process (read, modify and write). 

The Java common library provides a concurrent hash table (\emph{ConcurrentHashMap}) with concurrent control primitives  equivalent to the ones we include in our data store \gls{api}. 
However, the control is based on the logical equivalence of values instead of version numbers. 
That is to say, that instead of providing the version number in a conditional write, the client must provide the value that it expects to find in the hash table. 
Then, the hash table implementation can perform a logical test to assert if the client provided value is logically equivalent to the one that it holds. If so, the write is allowed, otherwise it is ``aborted''.  

While adapting existing applications to our \gls{api} that used the Java concurrent hash table, we have chosen to modify them slightly to use the version number mechanism instead of the existent logical equivalence. We do so for two reasons. First, while equivalence tests work well with objects, the same is not true for the raw bytes that result from the marshalling process (used to transform objects into byte arrays as required by our data store interface). In fact, we found cases where despite the  logical equivalence of two objects (as observed by their Java \texttt{equals} contract) their byte representation was disparate. Later, we have found that different object constructors for a list implementation resulted in different byte representations even when the list contained exactly the same objects. Truly, it can be painfully to the programmer to have control over this type of low level detail. Second, with versions numbers we can  reduce the \gls{rpc} message sizes (that are sent to the data store) significantly (see Section~\ref{sec:feasibility:apps}) since the version number is often much smaller than a value. 

This mechanism offers some control over concurrent updates to clients of the data store since the replace and remove operations that use an \texttt{expected\_version} argument (see in Fig~\ref{fig:design:class-diagram}). The data store executes those operations if, and only if, the client provided \texttt{expected\_version} matches the version of the data store for the particular key that is also provided in the operation.  
This the data store client with concurrent operations (e.g., \texttt{replace, remove}) that are only successful if the \texttt{expected\_version} provided by the client is equal to the version number that the data store has associated with the particular key that is also provided in the operation. 

Another common functionality is concurrency control based on version numbers that are associated with an entry in a key value table.
Every update operation done to an entry causes the data store to increment the version number. 
With this mechanism we enable a simple concurrency interface that allows different clients of the data store to manipulate it safely. 
This mechanism is commonly used to safely update a value.  
For example, imagine that a controller wishes to track for each website the set of adresses \gls{ip} addresses that perform requests to it. 
This information can be kept in a table that maps  an \gls{url} an \gls{ip}.  Fig~\ref{} TODO-FIG shows an example where two different controllers concurrently update this value. 
In the beginning (at the left) two controllers read the current value concurrently and both obtain the same result (a set containing \texttt{IP1}). Then later both controllers update this set and perform a write in the data store. In this situation the last write wins, which in the example belongs the bottom controller. 
As a result the \texttt{IP2} request to the website is not tracked. To avoid this behaviour both controller can use conditional updates based on versioning. 
Fig blah shows this example. This time, the response obtained from the data store includes a version number (1 for both clients). Then when they update the value they include the version number. In this case the first write (from the top client) is successful since the version number provided by the client matches the version number in the data store. But when the second client goes to write its  value then the data store does not allow it since the version number does not match the current version of that entry in the data store since it has been updated by the previous write. 
Concurrent updates could be avoided if the client provides both the value that he expects to find at the data store as well as the new value (the updated one). However this is not practical for two reasons. First in incurs in an significant overhead to the message sent to the data store (roughly the double of the size in most examples we have found). Second, and arguably more importantly, it is painfully to guarantee that byte array representations of marshalled values are actually equal even whey the values are logically equivalent. We have lost a lot of time tracking bugs caused by this when two different List constructor methods caused an object to have different byte representations even if the both lists contained exactly the same elements, and in the same order. 

 To avoid loosing values added concurrently to the set, we can use version numbers when re-inserting the set in the data store. This 
This mechanism is more commonly used in situations where we read a value, transform it and then update it. In order to not risk loosing the updates done by a concurrent client, we can use the version number of the original value when updating. The data store then can verify if the version is the least of the value or not. If not the update will fail, and we can proceed again.  

We found examples of applications (Load Balancer and Device Manager) that used this simple concurrency control primitives as defined by the Java \emph{ConcurrentMap} interface. However this interface is based on values instead of version numbers. This requires clients to send two values: one that should be equal to the data store version (only then will the operation succeed)  ; and the new one. 
In Java the concurrent hash table interfaces requires sending both the expected value and the new value. 
However with remote data stores this is not practical for two reasons. 
First, it incurs in a significant overhead since we have to send two values instead of one.
Second, it can be painfully to make sure that byte array representations of values are actually equal. When marshalling a value (transform an object into a linear byte array representation) the process in place can output different values which are logically equivalent (as specified by their equals contract in the Java case).  This can make a list that contains the same objects to have different representations which is something hard to identify and correct since we have to verify every suspected attribute of a class. 

So all in all the usage of replace based on byte comparison is not advocated. Instead timestamps end up benefiting the user by being more space efficient in the message exchanged (with a possibly insignificant cost for carrying the timestamp value); and by being easier to work with. 

\subsection{Columns}
\label{sec:heimdall:columns}
\begin{figure}[ht]
  \centering
  \begin{subfigure}[b]{0.5\textwidth}
                \centering
                \includegraphics[width=\textwidth]{./pic/heimdall/key-value}
                \caption{Key Value store.} 
                \label{fig:heimdall:columns-0}

        \end{subfigure}%
        ~
        \begin{subfigure}[b]{0.5\textwidth}
                \centering
                \includegraphics[width=\textwidth]{./pic/heimdall/column-store}
                \caption{Column store.}
                \label{fig:heimdall:columns-1}
        \end{subfigure}
  \caption{From a Key Value store to a Column store. }
  \label{fig:heimdall:columns} 
\end{figure}

With Columns we enhance the uni-dimensional model of a Key Value table  to a bi-dimensional one whereby two keys (as opposed to one) can access an individual attribute of a value inside a table.  

%Fig~\ref{fig:heimdall:columns} clarifies this transformation. 
With a Key Value data model (Fig.~\ref{fig:heimdall:columns-0}) clients are able to map an unique an unique key to any arbitrary value with no syntactical meaning for the data store (it is just raw data). 
This is a quite limited data model for the reason that values are often composed of multiple attributes. 
% with no syntactically meaning for the data store (it is just raw data). 
Ergo, we expanded the Key Value table (Fig~\ref{fig:heimdall:columns-0}) to allow clients to access the individual components of a value with an additional key (i.e., the column name). 


Despite the fact that a Column table decomposes a value into columns, the client is still able to manipulate the entire value.
In fact, the class diagram introduced before (Fig~\ref{fig:design:class-diagram}) shows that the client \gls{api} for a a \texttt{IColumnTable} inherits all the \texttt{IKeyValueTable} methods. 
Namely, the client is still able to retrieve or update a value  ``entirely'' (i.e., a Java object) even if he is not aware of  the column names that compose a value. 

Furthermore, the columns names are not static, not even in the context of  a table. Each key-value entry may have different columns as defined by the clients that can add and delete columns from a value as they see fit. 
Off course in the context of distributed access, clients should be made aware of the columns that compose a value. 
This can be done in several ways. 

In our own experience with the data store, we use Java Annotations to mark the objects attributes that should be kept in the table  as well as their names (i.e., columns). 
Then, with the help of Java Reflection\footnote{Java Reflection enables, among other things, dynamic (run time) method invocations in objects.} we were able to, dynamically and deterministically marshall and un-marshall an object to/from a column-value map.  


A column based table model is beneficial because reading an entire value introduces considerable overhead in the message returned by the data store (since the message is size is considerable big).
Even though the request size of the client-to-datastore \gls{rpc} message has a bigger performance impact in our data store (due the overhead introduced by the consensus algorithm performed by the BFT-SMaRt middleware) than the return message, we believed that some cases justified the use of columns since they allowed to decrease the message size. 
And indeed, we did, but this turn out to have no effect in our performance evaluation.
 
Nevertheless we were still able to leverage on the Column table to improve the request size of messages through the \texttt{replaceColumn} method  that replaces a column inside a value using the Versioning technique described previously. 
However, in practice there is room to improvement, since we only have one version number associated with each key-value entry despite the fact that values are decomposed in columns. 
Consequently, in the context of concurrent updates to different columns, clients will see their update aborted by the data store, when in fact, their update could be applied. 
To solve this, we just need to use, a different version number to each column. 

However, once again,  as discussed in Section~\ref{sec:feasibility:apps} ours results did not show any significant improvement, so the use of Column Store turn out to be a waste. It did not improve  the code simplicity nor the performance of the controllers applications. 
Furthermore, it adds significant complexity to the client when we consider caching (that will be introduced in the following sections) since the clients, can read partial values from the local cache. 

%Values can be composed of different attributes causing their space usage to be huge. In simple tests that we have performed we saw devices instances that would go up to 2kB. It may not seem much but remember that our middleware could support 20kOps/s with 1Kb and only 4.7kOps/s with 4kOps/s. 

%So it pays offs to be economical in sizes. With column enabled data stores we do this by selectively reading one or more object attributes. 
%The columns that compose an object are not static, but defined by clients. Clients can add and delete columns from a single data store value  at any time (in one table, different values may have different columns). There are no empty columns.  

%The basic idea is to minimize the size of messages by being able to selectiveley access the sub elements of data store values that are constantly accessed on a per flow basis. 
%As an example (elaborated on the Evaluation section) Load Balancer requires reading an \gls{ip} address for every \gls{of} addressed at a balanced resource. By using an columnar approach we were able to improve from 224 bytes to 4 . So with columns we can reduce communication  with the data store to the very minimal. 


\subsection{Micro Components}
\label{sec:heimdall:micro-components}
Micro Components are equivalent to the stored procedures functionality existent in \gls{sql} databases. 
In essence a micro component is an arbitrary long  method that is executed in the data store and triggered by the client. 

This method, has semantic knowledge of the data that is contained in the data store. 
That is to say that it knows what to do with the data kept in the data store, which implies that it is knowns the marshalling and un-marshalling process used for the tables that it manipulates.

The most significant advantage of a micro component is performance since they allow the data store client to merge consecutive data store interactions in a single method.
This dimishes the latency impact that the data store has in the client goal whenever this is composed of more than one data store requests. 
In fact, as we show later, along side with the Cross Reference functionality (which also reduces the number of messages in a client-to-data store interaction), micro components  introduced the most significant performance improvements in the evaluation of our data store. 

In out prototype, we  developed micro components by statically (i.e., prior to compilation) incorporating the code in the data store along with the required Classes that the method required to operate. 
This is undesirable, mainly because it forces the re-deployment of the data store code in order to add new functionalities for the clients. 

However, the implementation of an dynamic micro components framework is far from unfeasible.
For this we only require to have an adaptive \gls{rpc} implementation in the data store, as well as the capability to load new Classes in runtime to the data store. 
This are reasonable requirements, but lacking the time to do so, we use the statical micro components to evaluate the impact of this functionality. 
In the future, we plan to address this issue. 

\subsection{Cache}
\label{sec:heimdall:cache}
With a Cache table, the client enables caching for a particular data store table. 
Consequently each value that is read or written from and to the data store is added to the local cache. 
The client can then leverage on the cache to avoid the latency penalty of interacting with  the data store. 
Even though this may seem inconsistent (no pun intended!)  with our design philosophy, it is still a much stronger consistency model that of an eventually consistent data store for the subtle  difference that the client has control over the level of consistency that he is willing to accept in each data store operation. 

To clarify, our cache is not an implementation detail hidden beneath the data store interface. 
On the contrary, we explicitly provide the Cache interface (\emph{ICachedKeyValueTable} and \emph{ICachedColumnTable}) as a functional element of our design for which the client has absolute control. 
Namely, the client is able to define if he accepts a cached value as well as the bound on the  window of inconsistency that he is willing to tolerate. 
Off course, this relies on the principle that the client operates in an synchronous computation model. 
Otherwise clock errors, and undefined bounds on computations can result in a client obtaining a value outside the specified bound. 

In order to define the inconsistency bound, the client can, whenever fetching a value from table, use the \texttt{get} method in a \emph{ICachedKeyValueTable} that accepts an argument (\texttt{accepted\_staleness}) defining the upper time bound for accepting the value present in the cache. 
Then, for each client request, the cache returns the a local value if it has been added to the cache within the time specified by the client (i.e., the time passed between adding the value to the cache and the current time is lower than the bound). 
Otherwise, the cache retrieves the  value from the data store. 
It is worth pointing out than if the bound specified by the client is 0, then the cache must forcibly fetch the value from the data store (hence providing consistency). 

To be clear, caching values breaks the consistency semantics of our design. 
This is true, since the values present at the local cache may be outdated when compared to the data store versions. 
Even so, with caching we have a much stronger consistency design when compared to using an eventually consistent data store. 

First, clients have the freedom to choose whether they are willing to accept a, possibly stale value present in cache or a consistent value retrieved from the data store. 
Furthermore they have explicit control over the window of inconsistency that they are willing to accept. 
Second, clients still have a strong consistency data store on which they can rely upon to evaluate the consistency of their cached values. 
Third, as long as writes are performed consistently there is no risk of conflicting values. 
The same is not true for eventually consistent data stores. 

%First, clients are limited to the window of inconsistency present in the data store which is outside their control. 
%Second, clients can not use the data store to evaluate the consistency of their data since the data store does not has a strong consistency model. 
%Third, with eventual consistent data stores concurrent writes can result in conflicting values for which the dat

% First, the client is forced to 
% Indeed the local cache present at an host is a partial complete data store replica that is eventually consistent (as long as the client ends up fetching the data from the data store) and updated whenever the client fetches 
%  However, in contrast to using an eventually consistent 
% With a cache we can have a fine graine control over the staleness of data. 
% This means that we can choose if we want consistency or not.
% And if we choose the latter, we can define the window of inconsistency that we are willing to tolerate. 

% The cache implementation is simple. 
% Every time a value is saved to  cache (either by fetching it from the data store or by updating it in the data store) we associate the local time with that value. 
% Then the data store client can fetch a value in cache by specifying the accepted staleness of that value (e.g., 200 ms, 10 ms, etc., ). 
% Cross References values are also kept in cache as well as column values. But with column values we  keep partially completed objects in cache. For example a client of the data store that requests the device \gls{mac} address columns get a Device in return.  That Device only has the \gls{mac} attribute set. 

%%% Local Variables: 
%%% mode: latex
%%% TeX-master: "../PEI"
%%% End: 
