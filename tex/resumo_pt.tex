%\pagestyle{empty}

% ----------------------------------------------------------------------
% P�gina do resumo em Portugu�s:
\selectlanguage{portuguese}
\vspace*{2cm}
\begin{center} \Large \bf Resumo
\end{center}
\vspace*{1cm} \setlength{\baselineskip}{0.6cm}
\todo{Ignore...}
O sucesso da Internet é indiscutível. No entanto desde à muito tempo
que são feitas sérias críticas à sua arquitectura. 
Investigadores acreditam que o principal
problema reside no facto de os dispositivos de rede incorporarem 
funções distintas e complexas que vão além do objectivo de encaminhar pacotes para o qual foram criados. O melhor exemplo, são os protocolos distribuidos complexos que tanto os \emph{routers} como os \emph{switches} precisam de executar para as redes funcionarem correctamente. De forma a resolver este problema
uma arquitetura de rede diferente tem vindo a ser adoptada 
 tanto pela comunidade científica como pela indústria. Nestas novas redes, conhecidas como  \emph{Software Defined Networks} (SDN), há uma separação física entre o plano de controlo do plano de dados. Isto é, toda a lógica e estado de controlo da rede  é retirada dos dispositivos de rede, para passar a ser executada num \emph{controlador} \emph{logicamente} centralizado que com um visão global, lógica e coerente da rede, consegue controlar a mesma de forma dinâmica. Com esta delegação de funções para o controlador os dispositivos de rede podem dedicar-se exclusivamente à sua função essencial de encaminhar pacotes de dados. 
Embora existam controladores centralizados que escalam a milhares de \emph{hosts} há a necessidade de os tornar distribuido de forma a conseguir suportar a escala de \emph{data-centers} e redes modernas. No entanto a distribuição introduz vários desafios devido aos ambiente heterogêneo, assíncrono e faltoso onde os controladores necessariamente operam. 
Atualmente, os controladores de SDN distríbuidos utilizam modelos de distribuição não transparentes, com propriedades fracas como coerência futura que exigem maior cuidado no desenvolvimento de aplicações de controlo de rede no controlador. Isto deve-se à ideia mal-formada de que propriedades como coerência forte impedem a escalabilidade do controlador. 
O nosso trabalho irá contribuir com estudo e desenvolvimento dum controlador distribuido com coerência forte e tolerante a faltas para SDN. Este será conseguido com recurso a técnicas bem conhecidas de replicação baseada na máquina de estados distribuida. Um controlador com coerência forte traduz-se num modelo de programação mais simples e transparente à distribuição do controlador. Além disto, acreditámos que mesmo apresentando a propriedade de coerência forte, o nosso controlador irá conseguir apresentar uma performance superior à dos existentes na literatura. 

\todo{Pelo menos 1200 palavras} 
\vfill

\begin{flushleft}
\textbf{Palavras-chave:}
Replicação, Coerência Forte, Redes Controladas por Software, Tolerância a
Faltas, Máquina de Estados Distribuída, Plano de Controlo Distribuído. 
\end{flushleft}

\LIMPA
% Fim da p�gina do resumo em Portugu�s
% ----------------------------------------------------------------------
