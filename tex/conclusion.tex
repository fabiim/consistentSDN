
The introduction of a fault-tolerant, consistent data store in the architecture of a distributed SDN controller has a cost.
Adding fault tolerance increases the robustness of the system, while strong consistency facilitates application design, but the fact is that these mechanisms affect system performance.
First, the overall throughput will decrease to the least common denominator, which will in most settings be the data store.
Second, the total latency will increase as the response time for a data path request now has to include i) the latency to send a request to the data store; ii) the time to process the request; and iii) the latency to reply back to the controller.
Starting by assuming the inevitability of this cost, our objective in this paper is to show that, for some network applications at least, the cost may be bearable and the overall performance of the system remain acceptable.

Our argument is threefold.
First, we note that the performance results of our data store are similar to those reported for the original NOX and other popular SDN controllers\,\cite{Tootoonchian:2012:CPS:2228283.2228297}.
The average throughput for the Learning Switch application (the only application considered in\,\cite{Tootoonchian:2012:CPS:2228283.2228297}) is not far from that reported by NOX (30kReq/s), so our data store would not become a bottleneck in this respect.
In addition, the latency is close to the values reported for the different SDN controllers analyzed in that work (including the high-performance, multi-threaded ones), so the additional latency introduced, although non-negligible, can (arguably) be considered acceptable.
We consider this result to be remarkable given that our data store provides both strong consistency and fault tolerance.

Of course, the insightful reader will note that the results become quite distant from what is obtained with a controller that is optimized for performance, such as NOX-MT~\cite{Tootoonchian:2012:CPS:2228283.2228297}, particularly in terms of throughput.
As the second part of the argument, it is important to understand that every update to our data store represents an execution of the protocol of Fig. \ref{fig:paxos}, while in NOX-MT we have simply OF requests being received by a controller with the data store kept in main memory.
Even if NOX-MT (or any other high-performance controller) synchronously writes particular data to disk (something that takes around 5ms), no more than 200 updates/second can be executed.
This unequivocally shows that if some basic durability guarantees are required (e.g., to ensure recoverability after a crash), then the impressive capabilities of these high-performance controllers will be of little use.


In this paper we have proposed a distributed, highly-available, strongly consistent controller for SDNs.
The central element of the architecture is a fault-tolerant data store that guarantees acceptable performance.
We have studied the feasibility of this distributed controller by analyzing  the workloads generated by representative SDN applications and demonstrating that the data store is capable of handling these workloads, not becoming a system bottleneck.

The drawback of a strongly consistent, fault-tolerant approach for an SDN platform is the increase in latency, which limits responsiveness; and the decrease in throughput that hinders scalability.
Even assuming these negative consequences, an important conclusion of this study is that it is possible to achieve those goals while maintaining the performance penalty at an acceptable level.

As future work, we will focus on the optimization of the proposed distributed controller and on modifying the Floodlight applications to make them ``data store-aware'', as explained before. 
We plan to make heavy use of optimization techniques such as batching, caching and speculation to improve the data store considering the workload characteristics of SDN control applications.

As the number of SDN production networks increase the need for dependability becomes essential. The key takeover of this work is that dependability mechanisms have their cost, and it is therefore an interesting challenge for the SDN community to integrate these mechanisms into scalable control platforms. But, as argued in this paper, this is a challenge we, as a community, can surely meet.\\



\subsection{Limitations}
\begin{itemize}
\item Read entire tables, a common behaviour observed has not practical solution.  This will off course result in a giant overhead to the data store. Naturally this can be solved by introducing stale data with a cache, which may not be harmfull if the applications are not in a critical path to the network control goals (e.g., systems who display information). However we have seen this kind of code in Floodlight applications. In fact, there was one application that was left out of this work in a early stage, since after we adapt it to our data store and analyzed its behaviour we found that on each topological change to the network, it required reading all the data store state to behave 
\end{itemize}

\subsection{Future Work}
\begin{itemize}
\item Add applications that we believe to be fundamental to the data center: Topology Manager and Firewall.  Study the pipeline effect of all the applications together and improve on that. 
\item Study the differences between serializability and strong consistency, establishing when do the applications require one or another. State machine replication is not required in all aplications, use strong consistency in some cases (reads, writes). Bessani Protocol, cassandra with strong consistency scales better than SMR , etc., 
\end{itemize}