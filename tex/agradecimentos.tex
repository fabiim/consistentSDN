\pagestyle{plain}

\vspace*{2cm}
\begin{center}
\selectlanguage{portuguese}
\Large \bf Agradecimentos
%\selectlanguage{english}
%\Large \bf Acknowledgments
\end{center}
\vspace*{1cm} \setlength{\baselineskip}{0.6cm}

 Á minha familia, que mais do que ser familia, acreditou em mim, e levou-me às costas demasiados anos. Eu nunca vos vou conseguir retribuir o quanto fizeram por mim depois de tanto abismo cruzado, e de tanta rotunda demorada. As saídas nem sempre são óbvias à partida, independente da via de entrada escolhida. Pacientemente cheguei lá por  mim, mas nunca seria possível sem todos os vossos sacríficios e paciência. De seguida à minha  companheira de viagem, Maria Lalanda, por me acompanhar nesta jornada comprida. Foste o melhor deste caminho, espero no futuro perder-me apenas contigo. E nunca te esqueças que não chove em Santiago. 

Aos meus orientadores: Alysson,  Ramos e Diego Kreutz. Um muito obrigado por toda a paciência, apoio, e confiança ao longo do meu último ano. Sinto que estive muito acquém do que devido, por mais esforço que tenha colocado neste projecto. Espero ter aprendido com os erros, e espero que no restante processo desse trabalho, possa-vos supreender com mais e melhor. Também quero agradecer por ao prof. Alysson por me ter ensinado o \\vspace (prometo não sobre-utilizar :) e ao prof Fernando Ramos por me ter vendido o meu proprio peixe. 


Queria agradecer também aos reviewers da EWSDN pelos comentários que ajudaram a melhor o paper que foi resultado deste trabalho. Além disso queria agradacer ao financiamento que permitiu o meu trabalho ( parcialmente supportado pelo EC FP7 através do projecto BiobankCloud (ICT- 317871) e também pela FCT através do programa Multi-anual do LASIGE). 

De restante queria agradecer a uma grande lista de pessoas e coisas. Em especial: 
Ao pessoal do Jardim, pelas baldas, pelo jardim, por todo o tempo envoltos em ritmo e poesia. São demasiados os nomes para referir aqui, mas não podia deixar de referir as pessoas que por cirscunstância tiveram mais impacto em eu estar onde estou: João Sardinha, e Tâmara Andrade. Os restantes não deixam de ser igualmente importantes para mim. Á Catarina e Filipe Rebelo, e à senhora Eleanor por terem acreditado em mim.  Ao Manuel Barbosa por me alavancar em Braga, e ao  José Silva e à Barbara por se infiltrarem na minha casa :) .   Á nossa casa, e a voçês três por termos constituido familia. Nunca vou esquecer aquele espaço, apenas por causa da voçês. Á Senhora Conceição também. 


Aos Monads na janela, ao Pingo Doce, Lidl, Pizarria dos lux e a todas as máquinas automáticas do c5 e a do c6. Houvesse trocos em todas aquelas noitadas.... Ao café, ao SG Ventil, aos seguranças que estavam sempre em ronda e não me deixavam ir fumar : um não agradecimento. Ao gadgets do Tocha, ao quadro da residencia, à vista sobre a 25 de abrirl, e a todos os hipsters da príncipe real. Ao miradouro, ao jardim e ao quiosque. À Pizarria do Lux, a sério. Á máquina de café da residencia. Ao stack-overflow e ao lego-coding. 


% Ramos who taught how to sell fish. 
% Allyson who taught me about \vspace. I promise not to overuse it .  
%sá, we will eventually work again. 
%last but not least, to Leslie  Lamport, for wearing t-shirts on presentations,  inventing LaTex and a zillion of fundamental papers. 

%TO haskell who taught me to think. To java and c for making me unlearn that. God forbid (I will ever be a programmer!) 
Á longuria, ao insularismo e  á Ilha! 
E por último, mas não menos importante (olha o cliché!), ao Charlie que conseguiu sempre sair da selva. Mesmo sem cão. 

And last, but not least: to Charlie, who has always made out of the jungle! Event without a dog! 
\LIMPA
\LIMPA

~
\vfill

\selectlanguage{portuguese}
%\selectlanguage{english}
\begin{flushright}\textit{A todos ``os meus conhecidos''  que já levaram porrada. }\end{flushright}

\LIMPA

%%% Local Variables: 
%%% mode: latex
%%% TeX-master: "../PEI"
%%% End: 
