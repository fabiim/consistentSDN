\pagestyle{plain}

\vspace*{2cm}
\begin{center}
\selectlanguage{portuguese}
\Large \bf Agradecimentos
%\selectlanguage{english}
%\Large \bf Acknowledgments
\end{center}
\todo{Ignore...}
\vspace*{1cm} \setlength{\baselineskip}{0.6cm}
Aos meus orientadores --- Prof. Doutor Alysson Neves Bessani, Prof. Doutor Fernando Manuel Valente Ramos Doutor, e  Diego Kreutz, que mais do ocuparem titulos e posições tomaram lugar numa activa equipa de trabalho que tornou este projecto possivel. 
Um muito obrigado por toda a paciência, apoio, e confiança ao longo do meu último ano. 
No contexto do meu percurso na FCUL queria também agradecer ao Prof. Doutor Hugo Alexandre Tavares Miranda, que muito me ensinou sobre sistemas distribuídos, e também aos companheiros do laboratório 25 que muito me ensinaram e desafiaram. Este grupo de pessoas constitui o grupo de  pessoas mais inteligentes  e trabalhadoras que já tive oportunidade de conhecer. 
Queria agradecer também aos reviewers da EWSDN pelos comentários que ajudaram a melhorar o paper que foi resultado deste trabalho. Além disso queria agradacer ao financiamento que foi parcialmente supportado pelo EC FP7 através do projecto BiobankCloud (ICT-317871) e também pela FCT através do programa Multi-anual do LASIGE. 

No hemisférico pessoal queria agradecer a todos os que me carregaram até aqui. Nomeadamente a minha familia (pais e irmas), que mais do que ser familia, acreditou em mim, e levou-me às costas todos anos. Eu nunca vos vou conseguir retribuir o quanto fizeram por mim. 
De seguida à minha  companheira de viagem, Maria Lalanda, por me acompanhar nesta jornada comprida. Foste o melhor deste caminho, espero no futuro perder-me apenas contigo. 

De restante queria agradecer a uma grande lista de pessoas e coisas. Em especial: 
Ao pessoal do Jardim, pelas baldas, pelo jardim, por todo o tempo envoltos em ritmo e poesia. São demasiados os nomes para referir aqui, mas não podia deixar de referir as pessoas que mais me apoiaram neste percurso: João Sardinha, e Tâmara Andrade. 
Em Braga, à Catarina e Filipe Rebelo, e à senhora Eleanor por terem acreditado em mim.  Ao Manuel Barbosa por me alavancar, e ao  José Silva e à Barbara Manso por se infiltrarem na minha casa. Voçês três são familia (e a Conceição veio por arrasto :). Um especial agradecimento ao José, por ser o meu primeiro companheiro de batalha numa guerra bem complicada. Nunca é tarde para apanhar o comboio Cacilda! Aos Monads, à casa vintage, ao gadgets do Toxa, à bandeira que soprava na direcção de casa, e a todo o restante foklore. Ao JBB e JNO por inspirarem. 

%Sinto que estive muito acquém do que devido, por mais esforço que tenha colocado neste projecto. 
%Espero ter aprendido com os erros, e espero que no restante processo desse trabalho, possa-vos supreender com mais e melhor. 
%Também quero agradecer por ao prof. Alysson por me ter ensinado o \\vspace (prometo não sobre-utilizar :) e ao prof Fernando Ramos por me ter vendido o meu proprio peixe. 

Em Lisboa, à Filipa Costa e João Pereira, aos concertos e aos shots por suportarem o gajo mais chato de sempre. 
Ao MSI por ter me confrontado  com o maior desafio da minha vida. 
Ao José Lopes, porque sem ele nunca teria conseguido. Mesmo que conseguisse, não teria piada. Fico à espera que me ``faças um turing'' rapaz. 
Ao pessoal do FCUL que muito ajudou. Em especial o Emanuel, Juliana e Anderson. 
À RUMO que me acolheu. Com remorso não tive o impulso de explorar os teus cantos. 
À Maria e ao Zé pela casa e paciencia. Igualmente ao Miguel Costa, Francisco Apolinário . 
Finalmente a todas as máquinas automáticas do c* que me alimentaram durante as semanas do Globox, ao café, SG Ventil, stack-overflow e lego-coding: um muito obrigado. Ao Lamport por usar t-shirts em apresentações, inventar o \LaTeX e um zilião de papers. 
Por fim ao Charlie, ``who has always made it out of the jungle!''. Mesmo sem cão!\\ 

\emph{Á Lonjura, ao Insularismo e  á Ilha!} \\



\LIMPA
\LIMPA

~
\vfill

\selectlanguage{portuguese}
%\selectlanguage{english}
\begin{flushright}\textit{A todos ``os meus conhecidos''  que já levaram porrada. }\end{flushright}

\LIMPA

%%% Local Variables: 
%%% mode: latex
%%% TeX-master: "../PEI"
%%% End: 
