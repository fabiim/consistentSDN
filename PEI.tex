%
% Modelo para relat￳rio da disciplina de Projecto de Engenharia Informatica
% do MEI.
%
% Incorpora elementos impostos pelo Regulamento de Estudos Pos-Graduados da
% Universidade de Lisboa (Deliberacao 1506/2006 - Diario da Repblica, 2.a s←rie 
% - n.o 209 - 30 de Outubro de 2006)
%
\documentclass[12pt,openright,twoside]{report}

\usepackage[utf8]{inputenc}
% Quem tiver problemas com os acentos, trocar utf8 por latin1

\usepackage[portuguese,english]{babel}
\usepackage{times}
\usepackage{url}
\usepackage{graphicx}
\usepackage{mdwlist}
\usepackage[nottoc]{tocbibind}
\usepackage{csquotes}
% Indice remissivo
\usepackage{makeidx}
\makeindex

\usepackage{titlesec}

%\titlespacing{\subsubsection}{0pt}{0pt}{0pt}
\titleformat{\subsubsection}[runin]{\normalfont\bfseries}{\thesubsection.}{3pt}{}


% Glossario
\usepackage{glossaries}
\makeglossary

% Links
\usepackage{hyperref}

% Package para cabecalhos
\usepackage{fancyhdr}
\usepackage{lastpage}

\usepackage[font={scriptsize}]{caption}
\usepackage{subcaption}
\usepackage[sortcites=true, firstinits=true, isbn=false,
url=false, doi=false, eprint=false]{biblatex}

\bibliography{bibliografia,web}
\fancyhf{} %
\lhead{\nouppercase {\leftmark}} %
\rhead{\nouppercase {\bf \thepage}}
\renewcommand{\headrulewidth}{0.1pt}

% Comando para inserir pagina em branco (inserida na numeracao, mas sem
% numero impresso) para quando e' preciso obrigar um capitulo a comecar
% do lado direito (pagina impar)
\newcommand{\LIMPA}{
\newpage
\mbox{}
\thispagestyle{empty}
}

% Igual, mas insere pagina com numero impresso (normalmente nao se usa)
\newcommand{\LIMPAC}{
\newpage
\mbox{}
\thispagestyle{plain}
}


%%%%%%% PERSONAL COMMANDS %%%%%%%%%%%
\newcommand{\distcontrollers}{\cite{:vn, Tootoonchian:2010vy,Koponen:2010th,Yeganeh:2012jm}}
\newcommand{\distcontrollerspaper}{\cite{Tootoonchian:2010vy, Koponen:2010th,Yeganeh:2012jm}}
%END 


%
% ALTERAR AQUI AS INFORMACOES RELATIVAS AO PROJECTO
%
\newcommand{\PEITITULO}{A Consistent and Fault-Tolerant Network
  Information Base for Scalable Software Defined Networks}
\newcommand{\PEIAutor}{Fábio Andrade Botelho}
\newcommand{\PEIAutorNumAluno}{41625}

%Orientador e CoOrientador *sem* titulos (e.g. Prof. Doutor)
\newcommand{\PEIOrientador}{Alysson Neves Bessani}
\newcommand{\PEICoOrientador}{Fernando Manuel Valente Ramos} %se nao se aplicar, nao importa o que aqui esteja

%Se aplicavel, o supervisor pode ter um titulo (Dr., Eng.) colocado aqui
\newcommand{\PEISupervisorInstituicao}{Nome Completo do Supervisor}  %se nao se aplicar, nao importa o que aqui esteja

\newcommand{\PEIAnoLectivo}{2012/2013}
\newcommand{\PEIAno}{2013}

% Comentar/descomentar conforme conveniente
\newcommand{\PEITIPO}{DISSERTA\c{C}\~{A}O }
%\newcommand{\PEITIPO}{PROJECTO }

% Comentar/descomentar conforme conveniente
\newcommand{\PEIIdiomaTese}{\selectlanguage{portuguese}}
%\newcommand{\PEIIdiomaTese}{\selectlanguage{english}}

% Comentar/descomentar conforme conveniente
%\newcommand{\MEIEspecializacao}{Segurança Informática}
%\newcommand{\MEIEspecializacao}{Arquitectura, Sistemas e Redes de Computadores}
%\newcommand{\MEIEspecializacao}{Interac��o e Conhecimento}
%\newcommand{\MEIEspecializacao}{Engenharia de Software }

\usepackage{ifpdf}
\ifpdf
\pdfinfo {
	/Author (\PEIAutor)
	/Title (Projecto em Segurança Informatica)
	/Subject (Segurança Informatica)
	/Keywords (state machine replication, software defined networks)
	/CreationDate (D:20100510104905)
}
\fi

\usepackage[dvips]{geometry}
\geometry{a4paper=true,portrait=true,left=3cm,right=3cm,top=2.5cm,bottom=3.5cm}

\title{\PEITITULO}
\author{\PEIAutor}
%\date{\today}

\begin{document}

%Capa e pagina de rosto
\selectlanguage{portuguese}

\pagestyle{empty}

% ----------------------------------------------------------------------
% Capa
\begin{center}
\vspace{3cm}\normalfont\normalfont
\textsc{\huge{Universidade de Lisboa}}\\
\LARGE{Faculdade de Ci\^{e}ncias}\\
\Large{Departamento de Inform\'{a}tica}\\
\vspace{1cm}
\includegraphics[scale=.3]{pic/logo_ul.jpg}\\
%\includegraphics[scale=.6]{pic/logo_fcul.jpg}\\

\vspace{2cm}
\PEIIdiomaTese
\Large{\bf \PEITITULO}\\
\selectlanguage{portuguese}
\vspace{1cm}
%projecto realizado na\\
\vspace{0.7cm}
%\Large{\bf Nome da Instituicao de Acolhimento}\\
%\vspace{0.7cm}
%por\\
\vspace{0.7cm}
\Large{\bf \PEIAutor}\\
\vspace{2.4cm}
\Large{\bf \PEITIPO}\\
\vfill
\Large{MESTRADO EM SEGURANÇA  INFORM\'{A}TICA}\\

\vfill
\PEIAno

\end{center}
\newpage
\mbox{}
\newpage
% Fim da capa
% ----------------------------------------------------------------------

\setcounter{page}{1}
\pagenumbering{roman}

% ----------------------------------------------------------------------
% Folha de Rosto

\begin{center}
\vspace{3cm}\normalfont\normalfont
\textsc{\huge{Universidade de Lisboa}}\\
\LARGE{Faculdade de Ci\^{e}ncias}\\
\Large{Departamento de Inform\'{a}tica}\\
\vspace{0.9cm}
\includegraphics[scale=.3]{pic/logo_ul.jpg}\\
%\includegraphics[scale=.6]{pic/logo_fcul.jpg}\\
\vspace{1.9cm}
\PEIIdiomaTese
\Large{\bf \PEITITULO}\\
\selectlanguage{portuguese}
\vspace{1.4 cm}
\Large{\bf \PEIAutor}\\
%\vspace{2.4cm}
\vspace{1.9 cm}
\Large{\bf \PEITIPO}\\
\end{center}
\vspace{0.4 cm}
\begin{center}
\Large{MESTRADO EM SEGURANÇA INFORM\'{A}TICA}\\

\end{center}
\vspace{1 cm}
Disserta\c{c}\~{a}o orientada pelo Prof. Doutor \PEIOrientador \\
% DESCOMENTAR a linha relevante (se alguma), removendo o % no inicio
e co-orientado pelo Prof. Doutor \PEICoOrientador \\
%e por \PEISupervisorInstituicao
%\vfill
\vspace{-0.2cm} 
\begin{center}
\large\PEIAno
\end{center}
\newpage
\thispagestyle{empty}
\mbox{}
\newpage
% Fim da Folha de rosto
% ----------------------------------------------------------------------


% Agradecimentos
\pagestyle{plain}

\vspace*{2cm}
\begin{center}
\selectlanguage{portuguese}
\Large \bf Agradecimentos
%\selectlanguage{english}
%\Large \bf Acknowledgments
\end{center}
\vspace*{1cm} \setlength{\baselineskip}{0.6cm}
Aos meus orientadores --- Prof. Doutor Alysson Neves Bessani, Prof. Doutor Fernando Manuel Valente Ramos Doutor, e  Diego Kreutz, que mais do ocuparem titulos e posições tomaram lugar numa activa equipa de trabalho que tornou este projecto possivel. 
Um muito obrigado por toda a paciência, apoio, e confiança ao longo do meu último ano. 
No contexto do meu percurso na FCUL queria também agradecer ao Prof. Doutor Hugo Alexandre Tavares Miranda, que muito me ensinou sobre sistemas distribuídos, e também aos companheiros do laboratório 25 que muito me ensinaram e desafiaram. Este grupo de pessoas constitui o grupo de  pessoas mais inteligentes  e trabalhadoras que já tive oportunidade de conhecer. 
Queria agradecer também aos reviewers da EWSDN pelos comentários que ajudaram a melhorar o paper que foi resultado deste trabalho. Além disso queria agradacer ao financiamento que foi parcialmente supportado pelo EC FP7 através do projecto BiobankCloud (ICT-317871) e também pela FCT através do programa Multi-anual do LASIGE. 

No hemisférico pessoal queria agradecer a todos os que me carregaram até aqui. Nomeadamente a minha familia (pais e irmas), que mais do que ser familia, acreditou em mim, e levou-me às costas todos anos. Eu nunca vos vou conseguir retribuir o quanto fizeram por mim. 
De seguida à minha  companheira de viagem, Maria Lalanda, por me acompanhar nesta jornada comprida. Foste o melhor deste caminho, espero no futuro perder-me apenas contigo. 

De restante queria agradecer a uma grande lista de pessoas e coisas. Em especial: 
Ao pessoal do Jardim, pelas baldas, pelo jardim, por todo o tempo envoltos em ritmo e poesia. São demasiados os nomes para referir aqui, mas não podia deixar de referir as pessoas que mais me apoiaram neste percurso: João Sardinha, e Tâmara Andrade. 
Em Braga, à Catarina e Filipe Rebelo, e à senhora Eleanor por terem acreditado em mim.  Ao Manuel Barbosa por me alavancar, e ao  José Silva e à Barbara Manso por se infiltrarem na minha casa. Voçês três são familia (e a Conceição veio por arrasto :). Um especial agradecimento ao José, por ser o meu primeiro companheiro de batalha numa guerra bem complicada. Nunca é tarde para apanhar o comboio Cacilda! Aos Monads, à casa vintage, ao gadgets do Toxa, à bandeira que soprava na direcção de casa, e a todo o restante foklore. Ao JBB e JNO por inspirarem. 

%Sinto que estive muito acquém do que devido, por mais esforço que tenha colocado neste projecto. 
%Espero ter aprendido com os erros, e espero que no restante processo desse trabalho, possa-vos supreender com mais e melhor. 
%Também quero agradecer por ao prof. Alysson por me ter ensinado o \\vspace (prometo não sobre-utilizar :) e ao prof Fernando Ramos por me ter vendido o meu proprio peixe. 

Em Lisboa, à Filipa Costa e João Pereira, aos concertos e aos shots por suportarem o gajo mais chato de sempre. 
Ao MSI por ter me confrontado  com o maior desafio da minha vida. 
Ao José Lopes, porque sem ele nunca teria conseguido. Mesmo que conseguisse, não teria piada. Fico à espera que me ``faças um turing'' rapaz. 
Ao pessoal do FCUL que muito ajudou. Em especial o Emanuel, Juliana e Anderson. 
À RUMO que me acolheu. Com remorso não tive o impulso de explorar os teus cantos. 

Finalmente a todas as máquinas automáticas do c* que me alimentaram durante as semanas do Globox, ao café, SG Ventil, stack-overflow e lego-coding: um muito obrigado. Ao Lamport por usar t-shirts em apresentações, inventar o LaTeX e um zilião de papers. 
Por fim ao Charlie, ``who has always made it out of the jungle!''. Mesmo sem cão!\\ 

\emph{À Lonjura, ao Insularismo e  à Ilha!} \\



\LIMPA
\LIMPA

~
\vfill

\selectlanguage{portuguese}
%\selectlanguage{english}
\begin{flushright}\textit{A todos ``os meus conhecidos''  que já levaram porrada. }\end{flushright}

\LIMPA

%%% Local Variables: 
%%% mode: latex
%%% TeX-master: "../PEI"
%%% End: 


% Pagina do resumo em portugues
%\pagestyle{empty}

% ----------------------------------------------------------------------
% P�gina do resumo em Portugu�s:
\selectlanguage{portuguese}
\vspace*{2cm}
\begin{center} \Large \bf Resumo
\end{center}
\vspace*{1cm} \setlength{\baselineskip}{0.6cm}
\todo{Ignore...}
O sucesso da Internet é indiscutível. No entanto desde à muito tempo
que são feitas sérias críticas à sua arquitectura. 
Investigadores acreditam que o principal
problema reside no facto de os dispositivos de rede incorporarem 
funções distintas e complexas que vão além do objectivo de encaminhar pacotes para o qual foram criados. O melhor exemplo, são os protocolos distribuidos complexos que tanto os \emph{routers} como os \emph{switches} precisam de executar para as redes funcionarem correctamente. De forma a resolver este problema
uma arquitetura de rede diferente tem vindo a ser adoptada 
 tanto pela comunidade científica como pela indústria. Nestas novas redes, conhecidas como  \emph{Software Defined Networks} (SDN), há uma separação física entre o plano de controlo do plano de dados. Isto é, toda a lógica e estado de controlo da rede  é retirada dos dispositivos de rede, para passar a ser executada num \emph{controlador} \emph{logicamente} centralizado que com um visão global, lógica e coerente da rede, consegue controlar a mesma de forma dinâmica. Com esta delegação de funções para o controlador os dispositivos de rede podem dedicar-se exclusivamente à sua função essencial de encaminhar pacotes de dados. 
Embora existam controladores centralizados que escalam a milhares de \emph{hosts} há a necessidade de os tornar distribuido de forma a conseguir suportar a escala de \emph{data-centers} e redes modernas. No entanto a distribuição introduz vários desafios devido aos ambiente heterogêneo, assíncrono e faltoso onde os controladores necessariamente operam. 
Atualmente, os controladores de SDN distríbuidos utilizam modelos de distribuição não transparentes, com propriedades fracas como coerência futura que exigem maior cuidado no desenvolvimento de aplicações de controlo de rede no controlador. Isto deve-se à ideia mal-formada de que propriedades como coerência forte impedem a escalabilidade do controlador. 
O nosso trabalho irá contribuir com estudo e desenvolvimento dum controlador distribuido com coerência forte e tolerante a faltas para SDN. Este será conseguido com recurso a técnicas bem conhecidas de replicação baseada na máquina de estados distribuida. Um controlador com coerência forte traduz-se num modelo de programação mais simples e transparente à distribuição do controlador. Além disto, acreditámos que mesmo apresentando a propriedade de coerência forte, o nosso controlador irá conseguir apresentar uma performance superior à dos existentes na literatura. 

\todo{Pelo menos 1200 palavras} 
\vfill

\begin{flushleft}
\textbf{Palavras-chave:}
Replicação, Coerência Forte, Redes Controladas por Software, Tolerância a
Faltas, Máquina de Estados Distribuída, Plano de Controlo Distribuído. 
\end{flushleft}

\LIMPA
% Fim da p�gina do resumo em Portugu�s
% ----------------------------------------------------------------------


% Pagina do resumo em ingles
% ----------------------------------------------------------------------
% P�gina do resumo em Ingl�s:
\selectlanguage{english}
\vspace*{2cm}
\begin{center}
\Large \bf Abstract
\end{center}
\vspace*{1cm} \setlength{\baselineskip}{0.6cm}

This work is motivated by the 
emergent network architecture of Software
Defined Networks where the control of the network is extracted from the
network devices and delegated to a
server named controller that is responsible for dynamically
configuring the network devices present in the infrastructure. The
controller has the advantage of logically centralizing the network
state in contrast to the previous model where state was distributed
across the network devices. If spite of this logical centralization,
the control plane (where the controller operates)  must be distributed
in order to support the scale of modern datacenters and
networks. However, this distribution introduces several challenges due
to the heterogeneous, asynchronous and faulty environment where the
controller operates. Current distributed controllers
lack transparency due to the  eventual consistency
properties employed in the distribution of the controller. This results in a
complex programming model for the development of network control applications. This work will
contribute with a fault-tolerant distributed controller with strong
consistency properties that allows a more transparent distribution of
the control plane. Also we believe that even if favoring 
consistency, we will be able to provide superior performance results
that those available in the literature. 


% The success of the current Internet infrastructure and architecture is
% indisputable. However, it  has long suffered from negative
% criticism. The "classic'' Internet architecture suffers from serious
% drawbacks in the management field as the inherent complexity
% associated with the manual configuration of  distributed 
% network equipment can be too much to bare with only rudimentary
% mechanisms as command line interfaces and hand-made configuration
% scripts. Human error is known to be the most probable cause of network
% downtime. Additionally the lack of control abstractions  in
% networking  prevents the deployment of both new and modified control functions.

% These problems are cumulative as both network designers
% and administrators struggle to satisfy modern network
% requirements. Currently several control mechanisms are designed,
% deployed and configured in isolation of each other. Known examples
% are:intra domain routing; inter domain routing; access control; QoS services; packet inspection;load-balancing; and intrusion detection. These control mechanisms
% are deployed in network devices such as routers and switches with no
% support for their modification. 
% Thus, devices that should  be optimized for data forwarding end up bundled with
% heavyweight control features that must be configured independently. 

% Software Defined Networking (SDN) radically changes the current IP
% architecture by extracting the control function from the network
% devices and shifting it to a centralized service known as the control
% plane. This plane, with the help of a coherent view of the network state
% can dynamically configure the network devices.
% The network state
% is centralized in a datastore known as Network Information Base (NIB)
% or View. The NIB  can provide up
% to date connectivity, namespace information, and other physical or
% logical information capable of satisfying network control  objectives. 
%  The result configuration is the output
% of a function applied to the current network state. In SDNs networks devices are
% only responsible  for packet forwarding  as opposed  to the classic
% Internet architecture where devices also run distributed algorithms
% in order to perform self-configuration. 

% Centralized control planes have been reported to scale to tens of
% thousands of hosts. However the current
% state of the art implementations are still far from satisfying the
% requirements of modern datacenters and Wide Area Networks as the high traffic present in these networks may lead a
% centralized service to exhaustion. In addition latency issues presented
% in  WAN's covered by a centralized control planes may not
% be acceptable under the quality of service desired. Finally a single
% point of failure is not an usual option  as the failure of the control plane
% may compromise network availability. All this  strongly
% suggests the distribution of the control plane. However, the
% distribution processes must not affect the benefits of the
% SDN architecture even if operating in a heterogeneous, asynchronous and faulty
% environment.

% Although SDN has been considered a hot
% research topic the existing work does not covers distribution of the
% control plane extensively. Two designs are often-cited: HyperFlow \cite{Tootoonchian:2010vy} and
% Onix \cite{Koponen:2010th}. The latter replicates network events that cause changes to the NIB
% while the former replicates and distributes the NIB state. We consider both those works to have serious
% drawbacks as lack of distribution transparency and weak consistency
% properties leading  to a complex programming model for
% the development of network control applications. 

% In our work we plan  to contribute to the Software
% Defined Network field by introducing a strong consistent Network
% Information Base integrated in an open source SDN controller. Our NIB
% is replicated across controllers instances in a distributed state machine
% fashion that we implement with  well-known protocols to the  Distributed
% Systems field. We believe that the Network community has overlooked
% both the correctness and simplicity advantages of strong consistent datastores as
% well as their performance. We
% also belief that correctness problems can arise from the use of 
% weak consistency datastores if the application is not fully aware of
% the distribution guarantees given by the NIB. 


%Resumo at� \textbf{300} palavras. 

\vfill

\begin{flushleft}
\textbf{Keywords:}
Replication, Strong Consistency, Distributed State Machine,
Distributed Control Plane, Software Defined Networking. 
\end{flushleft}

\LIMPA
\selectlanguage{portuguese}
% Fim da p�gina do resumo em Ingl�s.
% ----------------------------------------------------------------------


\pagestyle{plain}

\PEIIdiomaTese

% Indice
\tableofcontents

\LIMPA

%Lista de figuras
\listoffigures

\addcontentsline {toc} {chapter} {Lista de Figuras}
\newpage
\thispagestyle{empty}
\mbox{}
\newpage

%Lista de tabelas
\listoftables

\addcontentsline {toc} {chapter} {Lista de Tabelas}
\newpage
\thispagestyle{empty} \mbox{}
\newpage

% ----------------------------------------------------------------------
% Inicio conteudo
\pagestyle{fancy}
\cleardoublepage

\setcounter{page}{1}
\pagenumbering{arabic}

% Conteudo (incl. Introducao)
\chapter{Introdução}

O relatório final deverá ter, em geral, entre 50 e 90 páginas (sem considerar anexos). O seu conteúdo deve realçar o trabalho realizado pelo aluno e a sua contribuição concreta no trabalho. Por exemplo, se o trabalho consiste no desenvolvimento de vários módulos a serem integrados num sistema mais global, o aluno deverá preocupar-se em descrever a parte que desenvolveu, como desenvolveu, que ferramentas usou, que alternativas poderiam existir, etc., em vez de efectuar uma descrição exaustiva das funcionalidades de todo o sistema.

O número de capítulos no relatório final não é rígido. No entanto, recomenda-se que sejam adoptados os seguintes princípios para a organização do relatório:

\begin{enumerate}
\item Um capítulo \emph{introdutório} no qual se apresentam o contexto do trabalho, se resume o trabalho desenvolvido, se identificam as contribuições deste e se apresenta a estrutura do próprio relatório. Deverá também ser mencionado sucintamente o enquadramento institucional em que o trabalho decorreu.
\item Um capítulo no qual se apresentam \emph{em pormenor} os \emph{objectivos} do trabalho, o \emph{contexto subjacente}, a \emph{metodologia} utilizada no seu desenvolvimento bem como o \emph{planeamento} efectuado para o concretizar. Deve também ser apresentada uma confrontação com o plano de trabalho inicial analisando as razões de eventuais desvios ocorridos.
\item Um capítulo onde é descrito o \emph{trabalho realizado}. Este é um dos capítulos fundamentais do relatório. Apresenta concretamente o que se fez de facto e as ferramentas usadas. De notar que é importante que fique claro qual a contribuição concreta do trabalho, sobretudo em casos de trabalho em equipa. Neste capítulo poderão ser inseridas questões relevantes da área de estudo em que o trabalho se integra, assim como o possível enquadramento num trabalho mais amplo. Eventualmente, e em função do âmbito e dimensão do trabalho, este capítulo poderá ser substituído por um conjunto de outros capítulos, que englobem em si o \emph{trabalho relacionado}, a \emph{análise} do problema, o \emph{desenho} da solução, a \emph{implementação} da solução e a \emph{avaliação} desta.
\item Um capítulo no qual são apresentadas as \emph{conclusões}. Para além de um \emph{sumário} do trabalho realizado, deve ser feito um \emph{comentário crítico} e serem apresentadas possibilidades de \emph{trabalho futuro} referindo o que falta fazer e o que poderá ser melhorado.
\item Um capítulo com a \emph{bibliografia} - lista de documentos usados e outras referências consideradas relevantes.
\item Um conjunto de capítulos com os \emph{anexos}. Quaisquer listagens, informação confidencial ou outras descrições muito pormenorizadas não devem ser integradas no corpo principal do relatório. Se houver necessidade de as apresentar, sugere-se a sua introdução em anexos ou em documentos separados. Os anexos suplementam o relatório e como tal devem ser referidos no corpo principal do relatório, descrevendo o tipo de informação que se detalha em anexo.
\end{enumerate}

Quando concluir a sua Dissertação, ou Relatório Final, o aluno deverá entregar, no Gabinete de Estudos Pós-Graduados da FCUL, o seguinte:

\begin{itemize}
\item Requerimento de admissão a provas de Mestrado (ver secção 4.2 do Guia de PEI);
\item 7 exemplares da Dissertação, ou Relatório Final, (encadernados de forma a que seja possível escrever na lombada - não utilizar argolas);
\item 7 Curricula Vitae;
\item 3 CDs com a Dissertação, ou Relatório Final, gravado em formato PDF;
\item Parecer do orientador do DI sobre a Dissertação, ou Relatório Final, (ver secção 4.2 do Guia de PEI).
\end{itemize}

O aluno deverá também submeter a versão final, em formato PDF, da Dissertação, ou Relatório Final, através do PEIpal. Este documento não deverá, em condições normais, exceder os 5MiB (se isso acontecer então deve ser revista a qualidade das imagens evitando a inclusão de bitmaps).

Se houver fundamentação adequada, os relatórios de trabalho poderão ser escritos em Inglês. Para isso o aluno deverá entregar no Gabinete de Estudos Pós-Graduados da FCUL:

\begin{itemize}
\item Um pedido dirigido ao Presidente do Conselho Científico da FCUL, fundamentando a necessidade da escrita do relatório em Inglês (ver secção 4.2 do Guia de PEI);
\item Um parecer do orientador indicando que concorda com o pedido do aluno e, eventualmente, apresentando argumentos adicionais (ver secção 4.2 do Guia de PEI).
\end{itemize}

Tendo sido aceite a escrita em Inglês do relatório de trabalho, este deverá conter um resumo adicional em Português de, pelo menos, 1200 palavras.

\section{Motivação}

Lorem \index{Lorem} ipsum dolor sit amet, consectetuer adipiscing elit, sed diam nonummy nibh euismod tincidunt ut laoreet dolore magna aliquam erat volutpat. Ut wisi enim ad minim veniam, quis nostrud exerci tation ullamcorper suscipit lobortis nisl ut aliquip ex ea commodo consequat. Duis autem vel eum iriure dolor in hendrerit in vulputate velit esse molestie consequat, vel illum dolore eu feugiat nulla facilisis at vero eros et accumsan et iusto odio dignissim qui blandit praesent luptatum zzril delenit augue duis dolore te feugait nulla facilisi. Nam liber tempor cum soluta nobis eleifend option congue nihil imperdiet doming id quod mazim placerat facer possim assum. Typi non habent claritatem insitam; est usus legentis in iis qui facit eorum claritatem. Investigationes demonstraverunt lectores legere me lius quod ii legunt saepius. Claritas est etiam processus dynamicus, qui sequitur mutationem consuetudium lectorum. Mirum est notare quam littera gothica, quam nunc putamus parum claram, anteposuerit litterarum formas humanitatis per seacula quarta decima et quinta decima. Eodem modo typi, qui nunc nobis videntur parum clari, fiant sollemnes in futurum. Lorem ipsum dolor sit amet, consectetuer adipiscing elit, sed diam nonummy nibh euismod tincidunt ut laoreet dolore magna aliquam erat volutpat. Ut wisi enim ad minim veniam, quis nostrud exerci tation ullamcorper suscipit lobortis nisl ut aliquip ex ea commodo consequat. Duis autem vel eum iriure dolor in hendrerit in vulputate velit esse molestie consequat, vel illum dolore eu feugiat nulla facilisis at vero eros et accumsan et iusto odio dignissim qui blandit praesent luptatum zzril delenit augue duis dolore te feugait nulla facilisi. Nam liber tempor cum soluta nobis eleifend option congue nihil imperdiet doming id quod mazim placerat facer possim assum. Typi non habent claritatem insitam; est usus legentis in iis qui facit eorum claritatem. Investigationes demonstraverunt lectores legere me lius quod ii legunt saepius.

\section{Objectivos}

Lorem ipsum dolor sit amet, consectetuer adipiscing elit, sed diam nonummy nibh euismod tincidunt ut laoreet dolore magna aliquam erat volutpat. Ut wisi enim ad minim veniam, quis nostrud exerci tation ullamcorper suscipit lobortis nisl ut aliquip ex ea commodo consequat. Duis autem vel eum iriure dolor in hendrerit in vulputate velit esse molestie consequat, vel illum dolore eu feugiat nulla facilisis at vero eros et accumsan et iusto odio dignissim qui blandit praesent luptatum zzril delenit augue duis dolore te feugait nulla facilisi. Nam liber tempor cum soluta nobis eleifend option congue nihil imperdiet doming id quod mazim placerat facer possim assum. Typi non habent claritatem insitam; est usus legentis in iis qui facit eorum claritatem. Investigationes demonstraverunt lectores legere me lius quod ii legunt saepius. Claritas est etiam processus dynamicus, qui sequitur mutationem consuetudium lectorum. Mirum est notare quam littera gothica, quam nunc putamus parum claram, anteposuerit litterarum formas humanitatis per seacula quarta decima et quinta decima. Eodem modo typi, qui nunc nobis videntur parum clari, fiant sollemnes in futurum. Lorem ipsum dolor sit amet, consectetuer adipiscing elit, sed diam nonummy nibh euismod tincidunt ut laoreet dolore magna aliquam erat volutpat. Ut wisi enim ad minim veniam, quis nostrud exerci tation ullamcorper suscipit lobortis nisl ut aliquip ex ea commodo consequat. Duis autem vel eum iriure dolor in hendrerit in vulputate velit esse molestie consequat, vel illum dolore eu feugiat nulla facilisis at vero eros et accumsan et iusto odio dignissim qui blandit praesent luptatum zzril delenit augue duis dolore te feugait nulla facilisi. Nam liber tempor cum soluta nobis eleifend option congue nihil imperdiet doming id quod mazim placerat facer possim assum. Typi non habent claritatem insitam; est usus legentis in iis qui facit eorum claritatem. Investigationes demonstraverunt lectores legere me lius quod ii legunt saepius.

\section{Contribuições}

Lorem ipsum dolor sit amet, consectetuer adipiscing elit, sed diam nonummy nibh euismod tincidunt ut laoreet dolore magna aliquam erat volutpat. Ut wisi enim ad minim veniam, quis nostrud exerci tation ullamcorper suscipit lobortis nisl ut aliquip ex ea commodo consequat. Duis autem vel eum iriure dolor in hendrerit in vulputate velit esse molestie consequat, vel illum dolore eu feugiat nulla facilisis at vero eros et accumsan et iusto odio dignissim qui blandit praesent luptatum zzril delenit augue duis dolore te feugait nulla facilisi. Nam liber tempor cum soluta nobis eleifend option congue nihil imperdiet doming id quod mazim placerat facer possim assum. Typi non habent claritatem insitam; est usus legentis in iis qui facit eorum claritatem. Investigationes demonstraverunt lectores legere me lius quod ii legunt saepius. Claritas est etiam processus dynamicus, qui sequitur mutationem consuetudium lectorum. Mirum est notare quam littera gothica, quam nunc putamus parum claram, anteposuerit litterarum formas humanitatis per seacula quarta decima et quinta decima. Eodem modo typi, qui nunc nobis videntur parum clari, fiant sollemnes in futurum. Lorem ipsum dolor sit amet, consectetuer adipiscing elit, sed diam nonummy nibh euismod tincidunt ut laoreet dolore magna aliquam erat volutpat. Ut wisi enim ad minim veniam, quis nostrud exerci tation ullamcorper suscipit lobortis nisl ut aliquip ex ea commodo consequat. Duis autem vel eum iriure dolor in hendrerit in vulputate velit esse molestie consequat, vel illum dolore eu feugiat nulla facilisis at vero eros et accumsan et iusto odio dignissim qui blandit praesent luptatum zzril delenit augue duis dolore te feugait nulla facilisi. Nam liber tempor cum soluta nobis eleifend option congue nihil imperdiet doming id quod mazim placerat facer possim assum. Typi non habent claritatem insitam; est usus legentis in iis qui facit eorum claritatem. Investigationes demonstraverunt lectores legere me lius quod ii legunt saepius.

\section{Estrutura do documento}

Este documento está organizado da seguinte forma:
\begin{itemize}
\item Capítulo 2 – AAA
\item Capítulo 3 – BBB
\end{itemize}

\chapter{Trabalho relacionado}
\LIMPA

\chapter{Análise}
\LIMPA

\chapter{Desenho}
\LIMPA

\chapter{Implementação}
\LIMPA

\chapter{Resultados}
\LIMPA


% Conclusao
% Conclusao

\chapter{Conclusão}

Duis semper. Cras posuere. Vivamus dolor. Vivamus odio odio, cursus posuere, facilisis sed, tempor eget, enim. Curabitur imperdiet lacinia eros. Pellentesque vitae velit. Donec venenatis pharetra nunc. Quisque lectus. Donec tincidunt eros in lacus. Sed quis nibh quis velit cursus hendrerit. Praesent condimentum. Donec nonummy eros nec sem.

Donec commodo molestie magna. Sed porttitor gravida leo. Morbi suscipit imperdiet arcu. Proin quam turpis, scelerisque ac, ullamcorper in, consectetuer vitae, nisl. Suspendisse iaculis cursus dui. Nam congue tincidunt leo. Class aptent taciti sociosqu ad litora torquent per conubia nostra, per inceptos hymenaeos. Etiam consectetuer, eros ac sollicitudin lobortis, orci erat fringilla mi, a placerat massa dui vel lectus. Nulla facilisi. Etiam iaculis nulla id lorem. Suspendisse in massa et turpis gravida consectetuer. Maecenas vel lectus. Curabitur tristique blandit tortor. Mauris nisl lacus, tincidunt interdum, feugiat rutrum, aliquet sit amet, massa. Vestibulum interdum scelerisque ante. Pellentesque pede.

In ut sapien. Cras dapibus blandit velit. Etiam non quam. Mauris hendrerit nulla pharetra mauris. Aliquam volutpat ullamcorper ante. Nam a nisi vitae ligula porta porttitor. Etiam urna. Nulla id ipsum. Vivamus justo metus, vestibulum non, suscipit ut, sagittis quis, ante. Aliquam et mauris. Duis a nibh vel leo tincidunt dictum. Suspendisse scelerisque auctor ante.


% Inicio apendices
\appendix




% Fim do conteudo
% ----------------------------------------------------------------------

% Glossario

\LIMPA

%
% Para actualizar o glossario, e' preciso correr o script ./fazindice
% e voltar a gerar o PDF
%
\newacronym{sdn}{SDN}{Software Defined Network}

\newacronym{nib}{NIB}{Network Information Base}

\newacronym{wan}{WAN}{Wide Area Network}

\newacronym{nos}{NOS}{Network Operating System}

\newacronym{os}{OS}{Operating System}

\newacronym{onf}{ONF}{Open Network Foundation}

\newacronym{of}{OF}{OpenFlow}

\newacronym{fifo}{FIFO}{First In,First Out}

\newacronym{smr}{SMR}{State Machine Replication}

\newacronym{arp}{ARP}{Address Resolution Protocol}
\newacronym{api}{API}{Application Programming Interface}
\newacronym{lru}{LRU}{Least Recently Used} 
\newacronym{ip}{IP}{Internet Protocol}
\newacronym{icmp}{ICMP}{Internet Control Message Protocol} 
\newacronym{ipv6}{IPv6}{Internet Protocol version 6}
\newacronym{vlan}{VLAN}{Virtual Local Area Network} 
\newacronym{mac}{MAC}{Media Access Control}


\newacronym{rest}{REST}{Representational State Transfer} 
\newacronym{vip}{VIP}{Virtual Enpoint Internet Protocol} 
\newacronym{rpc}{RPC}{Remote Procedure Call} 
\newacronym{sql}{SQL}{Structured Query Language} 

\newacronym{tcp}{TCP}{Transmission Control Protocol}
\newacronym{rcp}{RCP}{Routing Control Platform}
\newacronym{bgp}{BGP}{Border Gateway Protocol}
\newacronym{as}{AS}{Autonomous System}
\newacronym{ibgp}{iBGP}{interior Border Gateway Protocol}
\newacronym{url}{URL}{Uniform Resource Locator}
\newacronym{dhcp}{DHCP}{Dynamic Host Configuration Protocol}
\newacronym{http}{HTTP}{HyperText Transport Protocol} 
\renewcommand{\glossaryname}{Abreviaturas}
%\setglossary{glodelim={\noexpand}}
%\setglossary{glsnumformat=ignore}
\printglossary
\addcontentsline {toc} {chapter} {Abreviaturas}

% Bibliografia

% \LIMPA

 \printbibliography[title=References,heading=bibintoc]

% Indice remissivo
\LIMPA
\printindex
\addcontentsline {toc} {chapter} {\'{I}ndice}

\end{document}
